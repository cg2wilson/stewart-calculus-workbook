\documentclass[notes]{subfiles}
\begin{document}
	\chapter{Inverse Functions}
	\addcontentsline{toc}{section}{6.1 - Inverse Functions}
	\refstepcounter{section}
	\fancyhead[RO,LE]{\bfseries \nameref{cs61}} 
	\fancyhead[LO,RE]{\bfseries \small \currentname}
	\fancyfoot[C]{{}}
	\fancyfoot[RO,LE]{\large \thepage}	%Footer on Right \thepage is pagenumber
	\fancyfoot[LO,RE]{\large Chapter 6.1}
	
\section*{Inverse Functions}\label{cs61}
	\subsection*{Before Class}
	\addcontentsline{toc}{subsection}{Before Class}
	\subsubsection*{Inverse Functions \& Properties}
	\addcontentsline{toc}{subsubsection}{Inverse Functions \& Properties}
		\begin{ex}
			The table below gives the population $P(t)$ of a bacterial culture, $t$ hours after it is introduced to an agar-filled petri dish.
			\begin{center}
				\begin{tabular}{|c|c|c|c|c|c|c|c|c|c|}\hline
					$t$ \textbf{hours}		& 0	&1	&2	&3	&4	&5	&6	&7	&8 \\ \hline
					$N =P(t)$ \textbf{bacteria}	&150	&165	&182	&200	&220	&243	&267	&294	&324\\ \hline
				\end{tabular}
			\end{center}
				\vs{.5}
			The \emph{inverse function} $\inv{P}(N)$, gives the time elapsed since a bacterial culture was introduced to an agar-filled petri dish, when the population is $N$ bacteria.  Use this information to fill out the table below.
			\begin{center}
				\begin{tabular}{|c|P{.4in}|P{.4in}|P{.4in}|P{.4in}|P{.4in}|P{.4in}|P{.4in}|P{.4in}|P{.4in}|}\hline
					$N =P(t)$ \textbf{bacteria}	&150	&165	&182	&200	&220	&243	&267	&294	&324\\ \hline
					& & & & & & & &  &\\
					$t = \inv{P}(N)$ \textbf{hours}& & & & & & & &  & \\ 
					& & & & & & & &  &\\ \hline
				\end{tabular}
			\end{center}
		\end{ex}
			\vs{1}
			
		\showto{st}{
		\begin{defn}[One-to-one Function]
			A function $f$ is said to be \textbf{one-to-one} if \blank{3.7},\\[20pt] or in notation, \blank{4}.
		\end{defn}
		}
		
		\showto{ins}{
		\begin{defn}[One-to-one Function]
			A function $f$ is said to be \textbf{one-to-one} if \emph{it never takes on the same value twice}, or in notation, $f(x_1) \neq f(x_2)$ when $x_1\neq x_2$.
		\end{defn}
		}
			\newpage
		
		\showto{st}{
		\begin{thm}[Horizontal Line Test]
			\\[30pt]
		\end{thm}
		}
		
		\showto{ins}{
		\begin{thm}[Horizontal Line Test]
			A function is one-to-one if and only if no horizontal line intersects its graph more than once.
		\end{thm}
		}
		
		\begin{ex}
			Is $f(x) = x^5$ one-to-one?  Why or why not?
		\end{ex}
			\vs{1}
			
		\begin{ex}
			Is $f(x) = x^2$ one-to-one?  Why or why not?
		\end{ex}
			\vs{1}
			
		\begin{question}
			Let $f(x) = x^k$, where $k$ is an even number.  Using the previous exercise, do you think this function is one-to-one?  Why or why not?
		\end{question}
			\vs{.5}
			\newpage
			
		\showto{st}{
		\begin{defn}[Inverse Function]
			Let $f$ be a one-to-one function with domain $A$ and range $B$.  The \textbf{inverse function} is notated\\[15pt] $\inv{f}$, with domain \blank{.5} and range \blank{.5}.  The inverse function is defined by the equation\\[30pt]
		\end{defn}
		}
		
		\showto{ins}{
			\begin{defn}[Inverse Function]
			Let $f$ be a one-to-one function with domain $A$ and range $B$.  The \textbf{inverse function} is notated $\inv{f}$, with domain $B$ and range $A$.  The inverse function is defined by the equation $\inv{f}(y) = x\iff f(x) = y$ for any $y\in B$.
		\end{defn}
		}
		
		\showto{st}{
		\begin{rmk}[Domain and Range of Inverse Functions]
			\\[40pt]
		\end{rmk}
		}
		
		\showto{ins}{
		\begin{rmk}[Domain and Range of Inverse Functions]
			\begin{itemize}
				\item The domain of $\inv{f}$ is the range of $f$
				\item The range of $\inv{f}$ is the domain of $f$
			\end{itemize}
		\end{rmk}
		}
		
		\begin{rmk}[Notation Alert!]
			$\inv{f}$ is a special notation to indicate the \emph{function inverse}; you should not confuse this with the notation for the \emph{multiplicative inverse/reciprocal}, such as $\inv{x}$.  That is, 
			\begin{itemize}
				\item $\inv{f}(x)$ denotes the inverse of a function
				\item $\inv{x}$ denotes the multiplicative inverse of a variable, i.e. $\inv{x} = \dfrac{1}{x}$
			\end{itemize}
			The reciprocal of $f(x)$ is written as $\inv{[f(x)]}$.  \textbf{Notice the placement of the} $-1$.
		\end{rmk}
		
		\begin{ex}
			Use the table below to answer the questions.  If an answer does not exist, write DNE.
		\end{ex}\\
		\begin{minipage}{.3\textwidth}
			\begin{center}
				\begin{tabular}{|c|c|c|}\hline
					$x$	& $f(x)$& $g(x)$\\ \hline
					0	& 5		& 10\\ \hline
					1	& 8		& 7\\ \hline
					2	& $-1$	& 3 \\ \hline
					3	& 13		& 1\\ \hline
					4	& 5		& 9\\ \hline
					5	& 3		& $-2$\\ \hline
				\end{tabular}
			\end{center}
		\end{minipage}
		\begin{minipage}{.6\textwidth}
			\begin{multicols*}{2}
			\begin{enumerate}[(a)]
				\setlength\itemsep{60pt}
				\item $\inv{g}(3)$
				\item $\inv{f}(5)$
					\columnbreak
				\item $f(\inv{f}(13))$
				\item $(\inv{g}\circ \inv{f})(8)$
			\end{enumerate}
			\end{multicols*}
		\end{minipage}
		\newpage
		
		\showto{st}{
		\begin{rmk}[Cancellation Property]
			Let $f$ be a function with domain $A$ and range $B$, and let $\inv{f}$ be its inverse function.  Then, we have the following properties:
			\\ \\ \\ \\ \\ \\
		\end{rmk}
		}
		
		\showto{ins}{
		\begin{rmk}[Cancellation Property]
			Let $f$ be a function with domain $A$ and range $B$, and let $\inv{f}$ be its inverse function.  Then, we have the following properties:\\
			$\inv{f}(f(x)) = x$ for all $x\in A$\\
			$f(\inv{f}(y)) = y$ for all $y\in B$
		\end{rmk}
		}
		
		\begin{ex}
			If $f(x) = x^5$, what is $\inv{f}(x)$?  Use the cancellation properties to check your answer.
		\end{ex}
			\vs{.5}
			
		\begin{ex}
			Find the inverse function of $g(y) = y^3 - 3$.
		\end{ex}
			\vs{1}
			
		There is a graphical interpretation of algebraically finding an inverse:
			\vs{2}
			\newpage
			
	\subsection*{Pre-Class Activities}
	\addcontentsline{toc}{subsection}{Pre-Class Activities}
		\begin{ex}
			If $f(x) = x^5 + x^3 + x$, find $\inv{f}(3)$ and $f(\inv{f}(2))$.
		\end{ex}
			\vs{1}
			
		\begin{ex}
			Find the inverse formula for the function $f(x) = \dfrac{4x-1}{2x+3}$
		\end{ex}
			\vs{1}
			
		\begin{ex}
			Find the inverse formula for the function $f(x) = \dfrac{1-\sqrt{x}}{1+\sqrt{x}}$
		\end{ex}
			\vs{1}
		\newpage
		
	\subsection*{In Class}
	\addcontentsline{toc}{subsection}{In Class}
	\subsubsection*{Calculus of Inverse Functions}
	\addcontentsline{toc}{subsubsection}{Calculus of Inverse Functions}
		\begin{rmk}[Continuity of Inverses]
			If $f$ is a one-to-one continuous function defined on the interval $I$, then \blank{1.5}\\[15pt] \blank{5}.
		\end{rmk}
		
		\begin{question}
			If a one-to-one function $f$ is differentiable on the interval $I$, is it necessarily true that $\inv{f}$ is also differentiable?  
		\end{question}
			\vs{1}
			
		\showto{st}{
		\begin{rmk}[Derivative of Inverses (at a Point)]
			If $f$ is a one-to-one, differentiable function at $x = a$ with inverse function $\inv{f}$ and \\[15pt] \blank{2}, then the inverse function is differentiable at $a$ and \\ \\ \\ \\
		\end{rmk}
		}
		\showto{ins}{
		\begin{rmk}[Derivative of Inverses (at a Point)]
			If $f$ is a one-to-one, differentiable function at $x = a$ with inverse function $\inv{f}$ and $f'(\inv{f}(a))\neq 0$, then the inverse function is differentiable at $a$ and 
			\[(\inv{f})'(a) = \dfrac{1}{f'(\inv{f}(a))}\]
		\end{rmk}
		}
		
		\begin{pf}
		
		\end{pf}
			\vs{2}
			\newpage
		
		If we replace $a$ in the above formula with $x$, we get another formula for the derivative of the inverse:
		\showto{st}{
		\begin{rmk}[Derivative of Inverses (as a Function)]
			$ $\\ \\ 
		\end{rmk}
		}
		\showto{ins}{
		\begin{rmk}[Derivative of Inverses (as a Function)]
			\[(\inv{f})'(x) = \dfrac{1}{f'(\inv{f}(x))}\]
		\end{rmk}
		}
		\begin{ex}
			Let $f(x) = 3x - \sin x$.  Find $(\inv{f})'(0)$.
		\end{ex}
			\vs{1}
			
		\begin{ex}
			Let $g(x) = \sqrt{x-2}$ and $a = 2$.  
			\begin{enumerate}[(a)]
				\item Show that $g$ is one-to-one.
					\vs{.5}
					
				\item Find $(\inv{g})'(a)$ using the formula above.
					\vs{.5}
					
				\item Find $(\inv{g})'(x)$, and give its domain and range.
					\vs{1}
			\end{enumerate}
		\end{ex}
			\newpage
			
		\begin{ex}
			Let $h(x) = 2x^2-8x$.
			\begin{enumerate}[(a)]
				\item $h(x)$ is not one-to-one.  Sketch it and determine an interval on which it can be made one-to-one.  This is called the \emph{restricted domain}.
					\vs{1}
					
				\item Complete the square on $h(x)$ and use it to find the inverse function on your restricted domain.
					\vs{1}
					
				\item Find $(\inv{h})'(x)$ using your answer in (b).
					\vs{1}
					
				\item Find $(\inv{h})'(x)$ using formulas from this section.  Compare the two answers.
					\vs{1}
			\end{enumerate}
		\end{ex}
			\newpage
			
		\begin{ex}
			Find $(\inv{f})'(a)$ for the given functions:
			\begin{enumerate}[(a)]
				\item $f(x) = 3x^3 + 4x^2 + 6x + 5$, $a = 5$
					\vs{1}
					
				\item $f(x) = \sqrt{x^3+4x+4}$, $a = 3$
					\vs{1}
			\end{enumerate}
		\end{ex}
		
		\begin{ex}
			Suppose $\inv{f}$ is the inverse function of a differentiable function $f$ with $f(4) = 5$ and $f'(4) = \dfrac{2}{3}$.  Find $(\inv{f})'(5)$.
		\end{ex}
			\vs{1}
			
			\newpage
	\subsection*{After Class Activities}
	\addcontentsline{toc}{subsection}{After Class Activities}
		\begin{ex}
			Find $(\inv{f})'(2)$ for $f(x) = x^3 + 3\sin x + 2\cos x$
		\end{ex}
			\vs{1}
			
		\begin{ex}
			Suppose $\inv{f}$ is the inverse function of a differentiable function $f$, and let $G(x) = \dfrac{1}{\inv{f}(x)}$.  If $f(3) = 2$ and $f'(3) = \dfrac{1}{9}$, find $G'(2)$.
		\end{ex}
			\vs{1}
			
		\begin{ex}
			If $f(x) = \ds \int_3^x \sqrt{1+t^3}\, dt$, find $(\inv{f})'(0)$.
		\end{ex}
			\vs{1}
			
\clearpage
\end{document}
