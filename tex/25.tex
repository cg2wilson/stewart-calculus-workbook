\documentclass[notes]{subfiles}

\begin{document}
	\addcontentsline{toc}{section}{2.5 - The Chain Rule}
	\refstepcounter{section}
	\fancyhead[RO,LE]{\bfseries \large\nameref{cs25}} 
	\fancyhead[LO,RE]{\bfseries \currentname}
	\fancyfoot[C]{{}}
	\fancyfoot[RO,LE]{\large \thepage}	%Footer on Right \thepage is pagenumber
	\fancyfoot[LO,RE]{\large Chapter 2.5}
		
\section*{The Chain Rule}\label{cs25}
	\subsection*{Before Class}
	\addcontentsline{toc}{subsection}{Before Class}
	\subsubsection*{Review: Composition of Functions}
	\addcontentsline{toc}{subsubsection}{Review: Composition of Functions}
		\begin{ex}
			If $f(u)= \sqrt{u}$ and $u(x) = x^2 + 1$, find the composition $(f\circ u)(x)$.
		\end{ex}
			\vs{1}

		\begin{ex}
			If $f(x) = \sqrt{x}$ and $g(x) = \sqrt{2-x}$, find 
				\begin{enumerate}[(a)]
					\item $f\circ g$
						\vs{1}
						
					\item $g\circ f$
						\vs{1}
						
					\item $f\circ f$
						\vs{1}
						
					\item $g\circ g$
						\vs{1}
				\end{enumerate}
		\end{ex}
			\newpage
			
		\begin{ex}
			Let $k(x) = \cos 2x$.  Find $f,g$ such that $k(x) = f(g(x))$.
		\end{ex}
			\vs{1}
			
		\begin{ex}
			Let $f(x) = \sec^2 (x^2 + 9)$.  Find functions $a,b,c$ such that $f(x) = (a\circ b\circ c)(x)$
		\end{ex}
			\vs{1}
			
	\subsubsection*{The Chain Rule}
	\addcontentsline{toc}{subsubsection}{The Chain Rule}
		The chain rule \emph{relies} on being able to decompose a function into smaller pieces, and doing things in the right order.  
		\begin{thm}[Chain Rule]
			If $g$ is differentiable at $x$, and $f$ is differentiable at $g(x)$, then the composite function $F = f\circ g$ defined by $F(x) = f(g(x))$ is differentiable at $x$, and 
			\showto{ins}{
				\[F'(x) = f'(g(x))\cdot g'(x)\]
			}
			\showto{st}{
				\vspace{.75in} \\
			}
			In Leibniz notatation, if $y = f(u)$ and $u = g(x)$ are both differentiable functions, then
			\showto{ins}{
				\[\dfrac{dy}{dx} = \dfrac{dy}{du}\cdot \dfrac{du}{dx}\]
			}
			\showto{st}{
				\vspace{.75in}
			}
		\end{thm}
			\newpage

		\begin{ex}
			Let $k(x) = \cos 2x$.  Find $k'(x)$ using Example 2.5.3.
		\end{ex}
			\vs{1}
		
		\begin{ex}
			Let $g(x) = \sin 4x$.  Find $g'(x)$.
		\end{ex}
			\vs{1}
		\begin{ex}
			Let $f(x) = \sec^2(x^2+9)$.  Use Example 2.5.4 to find $f'(x)$.
		\end{ex}
			\vs{1}
			\newpage
	
	\subsubsection*{Pre-Class Activities}
	\addcontentsline{toc}{subsubsection}{Pre-Class Activities}
		\begin{ex}
			You are given a composite function.  Identify the inner function $u = g(x)$ and the outer function $y = f(u)$.
			\begin{enumerate}[(a)]
				\item $ \sqrt[3]{1+4x}$
					\vs{.5}
					
				\item $\sin (\cot x)$
					\vs{.5}
					
				\item $(5x^6 + 2x^3)^4$
					\vs{.5}
			\end{enumerate}
		\end{ex}
		
		\begin{ex}
			Set $h(x) = \sin (x^2)$.  Identify the inner function $u = g(x)$ and the outer function $y = f(u)$.  Then, find the derivative $h'(x)$.
		\end{ex}
			\vs{1}
			
		\begin{ex}
			Set $k(t) = \sin^2(t)$.  Identify the inner function $u = g(t)$ and the outer function $y = f(u)$.  Then, find the derivative $k'(t)$.
		\end{ex}
			\vs{1}
			\newpage
			
	\subsection*{In-Class}	
	\addcontentsline{toc}{subsection}{In-Class}
		\begin{question}
			If $h(x) = f(g(x))$, use the chain rule to write the derivative $h'(x)$ and $\dfrac{dh}{dx}$ in \emph{prime notation} and \emph{Leibniz notation}.
		\end{question}
			\vs{1}
			
		\begin{ex}
			Let $f(x) = \sqrt{x^2 + 1}$.  Find $f'(x)$.
		\end{ex}
			\vs{1.5}
			
		\begin{ex}
			Let $g(x) = \cos (x^2)$.  Find $g'(x)$.
		\end{ex}
			\vs{1.5}
			
		\begin{ex}
			Find the derivative of $k(t) = (2t+1)^5(t^3-t+1)^4$
		\end{ex}
			\vs{2}
			\newpage
			
		\begin{ex}
			Let $h(x) = \sin^2(\sqrt{x^2-1})$.  Find $h'(x)$.
		\end{ex}
			\vs{2}
			
		\begin{ex}
			Find the first derivative of $F(x) = (5x^5+2x^3)^4$
		\end{ex}
			\vs{1}
			
		\begin{ex}
			Find the first derivative of $h(t) = (2-\sin t)^{3/2}$
		\end{ex}
			\vs{1}
			
		\begin{ex}
			Find the first derivative of $y=\dfrac{1}{(\cos t + \tan t)^2}$
		\end{ex}
			\vs{1}
			\newpage
		\begin{ex}
			Find the first derivative of $h(\theta) = \tan (\theta^2\sin\theta)$
		\end{ex}
			\vs{1}
			
		\begin{ex}
			Find the first derivative of $y=\lrpar{\dfrac{1-\cos 2x}{1+\sin 2x}}^3$
		\end{ex}
			\vs{1}
			
		\begin{ex}
			Find the first derivative of $f(t) = \sqrt{t+\sqrt{t}}$
		\end{ex}
			\vs{1}
			\newpage
			
		\begin{ex}
			Find the first derivative of $r(x) = (x^2+1)^3(x^2+2)^6$
		\end{ex}
			\vs{1}
			
		\begin{ex}
			Find the first derivative of $y = \sqrt[5]{\dfrac{x}{x-1}}$
		\end{ex}
			\vs{1}
			
		\begin{ex}
			Find the first derivative of $z = \sqrt{\sin(1+x^2)}$
		\end{ex}
			\vs{1}
			\newpage
			
		\begin{ex}
			Find the first derivative of $A(t) = \dfrac{t^2}{\sqrt{t^3+1}}$
		\end{ex}
			\vs{1}
			
		\begin{ex}
			Find the first derivative of $f(x) = \cos^4(\tan^3(x))$
		\end{ex}
			\vs{2}
			
		\begin{ex}
			Find the first derivative of $y = \cos\sqrt{\sin(\tan\pi x)}$
		\end{ex}
			\vs{2}
			\newpage
			
		\begin{ex}
			Find an equation of the tangent line to the curve $y = \sqrt{1+x^3}$ at the point $(2,3)$
		\end{ex}
			\vs{1}
			
		\begin{ex}
			Let $f(x) = [g(\cos x)]^2$.  Write an expression for $f'(x)$.
		\end{ex}
			\vs{1}
			
		\begin{ex}
			If $h(x) = \sqrt{4+3f(x)}$, $f(1) = 7$, and $f'(1) = 4$, find $h'(1)$.
		\end{ex}
			\vs{1}
			
		\begin{ex}
			If $g(x) = \sqrt{f(x)}$, where $f$ is the function shown, evaluate $g'(3)$.\\
			\includegraphics{2.5fig1}
		\end{ex}
			\vs{.25}
			\newpage
			
	\subsection*{After Class}	
	\addcontentsline{toc}{subsection}{After Class}
		\begin{ex}
			Find the first and second derivatives of $y = \sin (\cos 4\theta)$
		\end{ex}
			\vs{2}
		\begin{ex}
			Find the first and second derivatives of $y = \dfrac{4x}{\sqrt{x+1}}$
		\end{ex}
			\vs{2}
		\begin{ex}
			Find $D^{35} \sin \pi x$
		\end{ex}
			\vs{1}
			\newpage
			
		\begin{ex}
			A table of values for $f,g,f'$, and $g'$ is given:
			\begin{center}
				\begin{tabular}{|c|c|c|c|c|}\hline
					$x$	& $f(x)$	& $g(x)$	& $f'(x)$	& $g'(x)$ \\ \hline
					1 & 3 & 2 & 4 & 6\\
					2 & 1 & 8 & 5 & 7\\
					3 & 7 & 2 & 7 & 9\\ \hline
				\end{tabular}
			\end{center}
			\begin{enumerate}[(a)]
				\item Find $h'(1)$, if $h(x) = f(g(x))$
					\vs{.5}
					
				\item Find $H'(1)$, if $H(x) = g(f(x))$
					\vs{.5}
			\end{enumerate}
		\end{ex}
		
		\begin{ex}
			Suppose $f$ is differentiable on $\R$, and $A$ is a real number.  Let $F(x) = f(x^A)$ and $G(x) = [f(x)]^A$.  Find expressions for $F'(x)$ and $G'(x)$.
		\end{ex}
			\vs{1}
			
		\begin{ex}
			The chain rule is often a source of confusion and frustration for students in (and beyond) calculus.  What do you think will help the chain rule stick out in your mind?
		\end{ex}
			\vs{1}
		
	\clearpage
\end{document}
