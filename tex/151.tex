\documentclass[notes]{subfiles}
\begin{document}
	\chapter{Multiple Integrals}
	\addcontentsline{toc}{section}{15.1 - Double Integrals Over Rectangles}
	\refstepcounter{section}
	\fancyhead[RO,LE]{\bfseries \nameref{cs151}} 
	\fancyhead[LO,RE]{\bfseries \small \currentname}
	\fancyfoot[C]{{}}
	\fancyfoot[RO,LE]{\large \thepage}	%Footer on Right \thepage is pagenumber
	\fancyfoot[LO,RE]{\large Chapter 15.1}

\section*{Double Integrals Over Rectangles}\label{cs151}
	\subsection*{Before Class}
	\addcontentsline{toc}{subsection}{Before Class}
	\subsubsection*{Review: The Definite Integral}
		\begin{question}
			Let $f(x)$ be a function defined on the interval $[a,b]$.
			\begin{enumerate}[(a)]
				\item Describe what the definite integral $\ds \int_a^b f(x)\, dx$ means.  
					\vs{1}
				\item Write the process used to define the definite integral.
					\vs{1}
				\item Geometrically, what are we using to find the integral?
					\vs{.5}
			\end{enumerate}
		\end{question}
			\newpage
			
	\subsubsection*{ Double Integrals}
		\begin{question}
			Now let $f(x,y)$ be a function defined on the rectangle $[a,b]\times [c,d]$.  How could we extend the definition of the definite integral to a function of two variables?
		\end{question}
			\vs{1}

		\begin{ex}
			This example will develop the definition of the definite integral for a function $f(x,y)$.  Let $f(x,y)$ be defined on the region $R$ given by
				\[R = [a,b]\times [c,d] = \lrbrace{(x,y)\in\R^2\mid a\leq x\leq b, c\leq y\leq d}\]and let $S$ be the solid that lies above $R$ and under the graph of $f$:
				\[S = \lrbrace{(x,y,z)\in\R^3\mid 0\leq z\leq f(x,y), (x,y)\in R}\]
				\begin{enumerate}[(a)]
					\item What kind of object can we use to approximate the \textit{volume} of $S$?  What sort of expression do we need to approximate the volume?
						\vs{2}
						
					\item Can we improve the approximation?  How?
						\vs{1}
						
					\item Is there a way of converting from an \emph{approximation} to an \emph{exact} answer?  How?
						\vs{.5}
						
					\item The definition of the double integral is then:
						\vs{1}
				\end{enumerate}
		\end{ex}
			\newpage
			
		\begin{ex}
			Approximate the volume of the solid that lies above the square $R = [0,3]\times[0,3]$ and below the surface $x=x^2+y^2$.
		\end{ex}
			\vs{1}
			
		\begin{ex}
			If $R = \lrbrace{(x,y)\mid -3\leq x\leq 3, -2\leq y\leq 2}$, then evaluate the integral $\ds \iint_R \sqrt{9-x^2}\,dA$ by interpreting it as a volume.
		\end{ex}
			\vs{1}
			\newpage
				
	\subsection*{Pre-Class Activities}
		\newpage
		
	\subsection*{In Class}
	\addcontentsline{toc}{subsection}{In Class}
	
	\subsubsection*{Midpoint Rule}
		\begin{defn}[Midpoint Rule (Double Integrals)]
			If $\bar{x_i}$ is the midpoint of the interval $[x_{i-1},x_i]$ and $\bar{y_j}$ is the midpoint of the interval $[y_{j-1},y_j]$, then we can approximate the double integral of $f(x,y)$ over the region $R$ by\\[30pt]
			
			where $\Delta A$ = \blank{2.5}
		\end{defn}
	
		\begin{ex}
			Approximate $\ds \iint_R (x-3y^2)\, dA$, where $R = \lrbrace{(x,y)\mid 0\leq x\leq 2, 1\leq y\leq 2}$ and $m=n=2$, using midpoints.
		\end{ex}
			\vs{1}
			\newpage
			
	\subsubsection*{Iterated Integrals}
		Now let's see how to evaluate double integrals.
		\begin{ex}
			Consider the function $f(x,y)$, defined and integrable on the rectangle $R = [a,b]\times[c,d]$.
			\begin{enumerate}[(a)]
				\item How can we make sense of the integral $\ds\int_c^d f(x,y)\, dy$?  Is it a necessarily a number?
					\vs{1}
				\item If we set $A(x) = \ds\int_c^d f(x,y)\, dy$, then what does $\ds \int_a^b A(x)\, dx$ do?
					\vs{1}
				\item Put it all together.  Interpret the expressions
					\begin{center}
						\begin{tabular}{cc}
							$\ds \int_a^b\int_c^d f(x,y)\, dx\,dy$ & $\ds \int_c^d\int_a^b f(x,y)\, dy\,dx$
						\end{tabular}
					\end{center}
					\vs{1}
			\end{enumerate}
		\end{ex}
	
		\begin{ex}
			Let $f(x,y) = x^2y^3$, defined on the rectangle $R = \lrbrace{(x,y)\mid 0\leq x\leq 3,1\leq y\leq 2}$. Evaluate the following:
			\begin{enumerate}[(a)]
				\item $\ds \int_0^3 \int_1^2 f(x,y)\,dy\,dx$
					\vs{1}
					\newpage
					
				\item $\ds \int_1^2\int_0^3 f(x,y)\,dx\,dy$
					\vs{1}
			\end{enumerate}
		\end{ex}
		
		\begin{ex}
			Let $g(x,y) = xe^y$, defined on the rectangle $[0,1]\times[-1,1]$.  Evaluate the following:
			\begin{enumerate}[(a)]
				\item $\ds \iint_R g(x,y)\,dx\,dy$
					\vs{1}
					
				\item $\ds \iint_R g(x,y)\,dy\,dx$
					\vs{1}
			\end{enumerate}
		\end{ex}
			\newpage
			
		\begin{question}
			In the previous two examples, you should see some similarities between both parts.  What are those similarities?
		\end{question}
			\vs{1}
			
		\begin{theorem}[Fubini]
			If $f$ is continuous on the rectangle
				\[R = \lrbrace{(x,y)\mid a\leq x\leq b, c\leq y\leq d}\]
			then\\[30pt]
			
		\end{theorem}
			
		\begin{ex}
			Evaluate the double integral $\ds \iint_R (y-3xy)\, dA$, where $R = \lrbrace{(x,y)\mid -1\leq x\leq 5,0\leq y\leq 1}$
		\end{ex}
			\vs{1}
			\newpage
			
		\begin{ex}
			Let $f(x,y) =y\cos (xy)$.
			\begin{enumerate}[(a)]
				\item Evaluate the double integral $\ds \int_0^{\pi/4}\int_0^{\pi/2} f(x,y)\, dy\,dx$.  What do you notice?
					\vs{2}
				\item Evaluate the double integral $\ds \int_0^{\pi/2}\int_0^{\pi/4} f(x,y)\, dx\, dy$.  What do you notice?
					\vs{.5}
			\end{enumerate}
		\end{ex}	
			\newpage
			
		\begin{ex}
			Find the volume of the solid bounded by the elliptic paraboloid $x^2 + 4y^2 + z = 16$, the planes $x = 2$ and $y = 2$, and the three coordinate planes.
		\end{ex}
			\vs{1}
			\newsec
	\subsection*{After Class Activities}
	\addcontentsline{toc}{After Class Activities}
		\begin{ex}
			Evaluate $\ds \iint_R \sqrt{5}\, dA$, where $R = \lrbrace{(x,y)\mid 2\leq x\leq 5,-1\leq y\leq 2}$
		\end{ex}
			\vs{1}
			\newpage
		\begin{ex}
			Evaluate $\ds \iint_R (13-6x)\, dA$, where $R =\lrbrace{(x,y)\mid -1\leq x\leq 1, -2\leq y\leq 2}$
		\end{ex}
			\vs{1}
			
		\begin{ex}
			Evaluate $\ds \int_1^4\int_0^2 (3xy^2-3y)\, dy\, dx$
		\end{ex}	
			\vs{1}
			
		\begin{ex}
			Evaluate $\ds \int_0^2\int_0^2 (2x+2y)^2\, dx\, dy$
		\end{ex}
			\vs{1}
	\clearpage
\end{document}