\documentclass[notes]{subfiles}
\begin{document}
	\addcontentsline{toc}{section}{11.9 - Representations of Functions as Power Series}
	\refstepcounter{section}
	\fancyhead[RO,LE]{\bfseries \nameref{cs119}} 
	\fancyhead[LO,RE]{\bfseries \small \currentname}
	\fancyfoot[C]{{}}
	\fancyfoot[RO,LE]{\large \thepage}	%Footer on Right \thepage is pagenumber
	\fancyfoot[LO,RE]{\large Chapter 11.9}
	
\section*{Functions as Power Series}\label{cs119}
	\subsection*{Before Class}
	\subsubsection*{Functions as Power Series}
		\begin{rmk}[Recall]
			\[\dfrac{1}{1-x} = 1 + x + x^2 + x^3 + \cdots = \sum_{n=0}^\infty x^n\]
			for $|x| < 1$
		\end{rmk}
		\begin{ex}
			Write the function $f(x) = \dfrac{1}{1+x^2}$ as a power series and find its interval of convergence.
		\end{ex}
			\vs{1}
		
		\begin{ex}
			Find a power series representation for $\dfrac{1}{x + 5}$.
		\end{ex}
			\vs{1}
				
		\begin{ex}
			Find a power series representation for the function $g(x) = \dfrac{x^4}{x + 5}$.
		\end{ex}
			\vs{1}
			\newpage
			
	\subsection*{Pre-Class Activities}
		\begin{ex}
			Use this space to write any questions you might have from the videos.
		\end{ex}
			\vs{.5}
			
		\begin{ex}
			Find a power series representation for the function, and determine the interval of convergence.
			\begin{enumerate}[(a)]
				\item $f(x) = \dfrac{5}{1-4x^2}$
					\vs{1}
					
				\item $g(x) = \dfrac{4}{2x+3}$
					\vs{1}
					
				\item $h(x) = \dfrac{x - 1}{x+2}$
					\vs{1}
			\end{enumerate}
		\end{ex}
			\newpage
			
	\subsection*{In Class}
		\begin{ex}
			Express the function $f(x) = \dfrac{2x-4}{x^2-4x+3}$ as a power series, and find the interval of convergence.
		\end{ex}
			\vs{1}
			
	\subsubsection*{Calculus and Power Series}
		\begin{thm}[Differentiation \& Integration of Power Series]
			If the power series $\ds \sum c_n(x-a)^n$ has radius of convergence $R > 0$, then the function\\[15pt] $f(x) = \ds \sum_{n=0}^\infty c_n(x-a)^n$ is differentiable on the interval \blank{2.5}, and\\
			\begin{itemize}
				\setlength\itemsep{20pt}
				\item $f'(x) = $
				\item $\ds \int f(x)\, dx = $
			\end{itemize}$ $\\[15pt]
			
			The radii of convergence of both of these power series is \blank{1}.
		\end{thm}
		
		\begin{rmk}[Restatement of the Theorem]
			We can rewrite the conclusion of the above theorem as\\
			\begin{itemize}
				\setlength\itemsep{15pt}
				\item $\dfrac{d}{dx}\left[\ds \sum_{n=0}^\infty c_n(x-a)^n \right] = $
				\item $\ds \int \left[\ds \sum_{n=0}^\infty c_n(x-a)^n\right] = $
			\end{itemize}
		\end{rmk}
			\newpage
			
		\begin{ex}
			Express the function $f(x) = \dfrac{1}{(1-x)^2}$ in terms of a power series, and find its interval of convergence.
		\end{ex}
			\vs{1}
			
		\begin{ex}
			Find a power series representation for the function $\ln (1 + x)$ as well as its interval of convergence.
		\end{ex}
			\vs{1}
			
		\begin{ex}
			Show that $\arctan x = \ds \sum_{n=0}^\infty (-1)^n \dfrac{x^{2n+1}}{2n+1}$.
		\end{ex}
			\vs{1}
			\newpage
			
		\begin{ex}
			Find a power series representation for the function, and determine the radius of convergence.
			\begin{enumerate}[(a)]
				\item $f(x) = x^2 \inv{\tan}(x^3)$
					\vs{1}
					
				\item $g(x) = \dfrac{x}{(1+4x)^2}$
					\vs{1}
				
				\item $h(x) = \dfrac{1+x}{(1-x)^2}$
					\vs{1}
			\end{enumerate}
		\end{ex}
			\newpage
			
		\begin{ex}
			Evaluate the integral $\ds \int \dfrac{1}{1+t^3}\, dt$ as a power series.  what's its radius of convergence?
		\end{ex}
			\vs{1}
			
		\begin{ex}
			The Bessel function of order 1 is defiend to be $J_1(x) = \ds \sum_{n=0}^\infty \dfrac{(-1)^nx^{2n+1}}{n!(n+1)!2^{2n+1}}$.
			\begin{enumerate}[(a)]
				\item Show that $J_1$ satisfies the differential equation $x^2J_1''(x) + xJ_1'(x) + (x^2-1)J_1(x) = 0$.
					\vs{2}
					
				\item Show that $J_0'(x) = -J_1(x)$.  Recall that $J_0 = \ds \sum_{n=0}^\infty \dfrac{(-1)^nx^{2n}}{2^{2n}(n!)^2}$.
					\vs{1}
			\end{enumerate}
		\end{ex}
			\newpage
			
	\subsection*{After Class Activities}
		\begin{ex}
			Find a power series representation for $f(x) = \dfrac{x+a}{x^2 + a^2}$, for $a > 0$	
		\end{ex}
			\vs{1}
			
		\begin{ex}
			Write a power series representation for the following:
			\begin{enumerate}[(a)]
				\item $f(x) = \dfrac{2x+3}{x^2+3x+2}$
					\vs{1}
					
				\item $f(x) = \ln (5-x)$
					\vs{1}
					
				\item $g(x) = \dfrac{x^2 + x}{(1-x)^3}$
					\vs{1}
			\end{enumerate}
		\end{ex}
\clearpage
\end{document}