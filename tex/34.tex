\documentclass[notes]{subfiles}

\begin{document}
	\addcontentsline{toc}{section}{3.4 - Limits at Infinity \& Horizontal Asymptotes}
	\refstepcounter{section}
	\fancyhead[RO,LE]{\bfseries \large\nameref{cs34}} 
	\fancyhead[LO,RE]{\bfseries \currentname}
	\fancyfoot[C]{{}}
	\fancyfoot[RO,LE]{\large \thepage}	%Footer on Right \thepage is pagenumber
	\fancyfoot[LO,RE]{\large Chapter 3.4}
	
\section*{Limits at Infinity \& Horizontal Asymptotes}\label{cs34}
	\subsection*{Before Class}
	\addcontentsline{toc}{subsection}{Before Class}
	\subsubsection*{The Ideas}	
	\addcontentsline{toc}{subsubsection}{The Ideas}
		\begin{ex}
			Sketch the graph of $f(x) = \dfrac{x^2-1}{x^2+1}$ using the techniques of \S3.3.
		\end{ex}
			\vs{2}
			
		The graph should look like this:
		\begin{center}
			\begin{tikzpicture}
				\begin{axis}[
					height = 3in,
					width = .8\textwidth,
					every tick label/.append style={font=\small},
					axis x line = middle,
					axis y line = middle,
		    			every axis y label/.style={at={(ticklabel cs:1.15)}},
		    			%ytick = {-4, -2, -3, -1, 1, 2, 3, 4},
					y label style={at={(axis description cs:.5,1.15)},anchor=north},
		    			ylabel = {$f(x)$},
	    				every axis x label/.style= {at ={(ticklabel cs:1)}},
	    				%xtick = {-4,-3,-2,-1,1,2,3,4},
	    				x label style={at={(axis description cs:1.05,.5)},anchor=east},
	    				xlabel = {$x$},
	    				xmin = -10, xmax = 10,
	    				ymin = -1.5,ymax = 1.5
					]
					
					\showto{ins}{
						\addplot[<->,smooth, samples = 100, thick, domain = -9.5:9.5] {(x^2-1)/(x^2+1)};
					}
				\end{axis}
				
			\end{tikzpicture}
		\end{center}
			\newpage
			
		\begin{defn}[Limits at Infinity]
			Let $f$ be a function defined on some interval $(a,\infty)$.  Then
				\showto{ins}{
					\[\lim_{x\to\infty} f(x) = L\]
				}
				\showto{st}{
					\\[40pt] $ $ \\
				}
			means that the values of $f(x)$ can be made arbitrarily close to $L$ by requiring $x$ to be sufficiently large.  If $g$ is defined on some interval $(-\infty,a)$, then
				\showto{ins}{
					\[\lim_{x\to-\infty} g(x) = L\]
				}
				\showto{st}{
					\\[40pt] $ $ \\
				}
			means that the values of $g(x)$ can be made arbitrarily close to $L$ by requiring $x$ to be sufficiently large negative.
		\end{defn}
			
		We read the limits above (for $f$) as
		\showto{ins}{
			\begin{itemize}
				\item the limit of $f(x)$, as $x$ approaches $\infty$, is $L$
				\item the limit of $f(x)$ as $x$ increases without bound, is $L$
			\end{itemize}
		}
		\showto{st}{
			\begin{itemize}
			\setlength\itemsep{25pt}
				\item 
				\item 
			\end{itemize}
			\vspace*{.1in}
		}
		with the obvious changes for $g$.  
		\begin{defn}[Horizontal Asymptote]
			The line $y = L$ is called a \textbf{horizontal asymptote} of the curve $y = f(x)$ if either 
				\showto{ins}{
					\[\lim_{x\to\infty} f(x) = L\qquad \text{or}\qquad \lim_{x\to -\infty} f(x) = L\]
				}
				\showto{st}{
					\vspace{.75in}
				}
		\end{defn}
		
		\begin{ex}
			Write the horizontal asymptotes of the function $f(x) = \dfrac{x^2-1}{x^2 + 1}$
		\end{ex}
			\vs{1}
			\newpage
			
	\subsubsection*{Pre-Class Activities}	
	\addcontentsline{toc}{subsubsection}{Pre-Class Activities}
		\begin{ex}
			Write the horizontal asymptotes of the function  $f(x) =\dfrac{3x-2}{2x+1}$.
		\end{ex}
			\vs{1}
			
		\begin{ex}
			Does the function $f(x) = \dfrac{1-x^2}{x^3-x+1}$ have any horizontal asymptotes?  If it does, give their equation.  If it doesn't, explain why.
		\end{ex}
			\vs{1}
			
		\begin{ex}
			The function $f(x) = \dfrac{x-9}{\sqrt{4x^2 + 3}}$ has two horizontal asymptotes: $L = \dfrac{1}{2}$ and $L = - \dfrac{1}{2}$.  Use limit notation to describe the horizontal asymptotes.
		\end{ex}
			\vs{1}
			\newpage
	
	\subsection*{In-Class}		
	\addcontentsline{toc}{subsection}{In-Class}
	\subsubsection*{Computing Limits at Infinity}
	\addcontentsline{toc}{subsubsection}{Computing Limits at Infinity}
		\begin{question}
			Think about $\ds \lim_{x\to\infty} \dfrac{1}{x}$ and $\ds \lim_{x\to -\infty} \dfrac{1}{x}$.  What do you expect these limits to be?  Why?  What\\[5pt] about $\ds \lim_{x\to \pm \infty} x^r$, for some $r > 0$?
		\end{question}
			\vs{1}
			
		\begin{rmk}[Theorem]
			If $r > 0$ is a rational number, then 
				\showto{ins}{
					\[\lim_{x\to \infty} \dfrac{1}{x^r} = 0\]
				}
				\showto{st}{
					\vspace{.75in} \\ \\
				}
			If $r > 0$ is a rational number such that $x^r$ is defined for all $x$, then
				\showto{ins}{
					\[\lim_{x\to -\infty} \dfrac{1}{x^r} = 0\]
				}
				\showto{st}{
					\vspace{.75in}
				}
		\end{rmk}
		
		\begin{ex}
			Evaluate $\ds \lim_{x\to \infty} \dfrac{3x^2-x-2}{5x^2+4x+1}$
		\end{ex}
			\vs{1.5}
			\newpage
			
		\begin{ex}
			Find the asymptotes of $f(x) = \dfrac{\sqrt{2x^2+1}}{3x-5}$
		\end{ex}
			\vs{1}
			
		\begin{ex}
			Compute $\ds \lim_{x\to \infty} (\sqrt{x^2+2}-x)$
		\end{ex}
			\vs{1}
			\newpage
		\begin{ex}
			Find the following limits, or argue why it doesn't exist:
			\begin{enumerate}[(a)]
				\item $\ds \lim_{x\to -\infty} \dfrac{4x^3 + 6x^2 - 2}{2x^3 - 4x + 5}$
					\vs{1}
					
				\item $\ds \lim_{x\to -\infty} \dfrac{\sqrt{1+4x^6}}{2-x^3}$
					\vs{1}
					
				\item $\ds \lim_{x\to \infty} \cos x$
					\vs{1}
					
				\item $\ds \lim_{x\to \infty} \lrpar{\sqrt{9x^2 + x} - 3x}$
					\vs{1}
					
				\item $\ds \lim_{x\to \infty} \sqrt{x^2 + 2}$
					\vs{1}
			\end{enumerate}
		\end{ex}		
			\newpage
				
		\begin{ex}
			A function $f$ is a ratio of quadratic functions and has a vertical asymptote $x=4$ and just one $x-$intercept, $x=1$.  We know that $f$ has a removable discontinuity at $x=-1$, and that $\ds \lim_{x\to -1} f(x) = 2$.  Evaluate $f(0)$ and find any horizontal asymptotes of $f$.
		\end{ex}
			\vs{1}
			
		\begin{ex}
			Sketch the function $y = \dfrac{1 + 2x^2}{1+x^2}$ using the methods of \S3.3 and this section.
		\end{ex}		
			\vs{1}

			\newpage
			
	\subsection*{After Class}
	\addcontentsline{toc}{subsection}{After Class}		
		
			
		\begin{ex}
			Sketch the graph of a function that satisfies the conditions: $f(1) = f'(1) = 0$, $\ds \lim_{x\to 2^+} f(x) = \infty$, $\ds \lim_{x\to 2^-} f(x) = -\infty$, $\ds \lim_{x\to 0} f(x) = -\infty$, $\ds \lim_{x\to -\infty} = \infty$, $\ds \lim_{x\to \infty} f(x) = 0$, $f''(x) > 0$ for $x > 2$, $f''(x) < 0$ for $x < 0$ and for $0 < x < 2$.
		\end{ex}
			\vs{1}
			
		\begin{ex}
			Find $\ds \lim_{x\to \infty} f(x)$, if $\dfrac{3x - 1}{x} < f(x) < \dfrac{3x^2 + 6}{x^2}$ for all $x > 6$
		\end{ex}	
			\vs{1}

		\begin{ex}
			A tank contains 5000 L of pure water.  Brine containing 30 g of salt per liter of water is pumped into the tank at a rate of 25 L/min.  Write an expression for the concentration of salt after $t$ minutes (in grams per liter).  What happens to the concentration as $t\to\infty$?
		\end{ex}
			\vs{1}
	\clearpage
\end{document}