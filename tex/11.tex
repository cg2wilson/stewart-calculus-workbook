\documentclass[notes]{subfiles}
\begin{document}
	\chapter{Functions and Limits}
	\addcontentsline{toc}{section}{1.1 - Four Ways to Represent a Function}
	\setcounter{section}{1}
	\setcounter{page}{1}
	\fancyhead[RO,LE]{\bfseries \large \nameref{cs11}} 
	\fancyhead[LO,RE]{\bfseries \currentname}
	\fancyfoot[C]{{}}
	\fancyfoot[RO,LE]{\large \thepage}	%Footer on Right \thepage is pagenumber
	\fancyfoot[LO,RE]{\large Chapter 1.1}

\section*{Four Ways to Represent a Function}\label{cs11}
	\subsection*{Before Class}
	\addcontentsline{toc}{subsection}{Before Class}
	\subsubsection*{Functions}
	\addcontentsline{toc}{subsubsection}{Functions}
		A \textbf{relation} is a rule which links an \textbf{input} variable to an \textbf{output}; given one piece of information, we can determine the corresponding piece.  A special type of relation is one called a function.
			\vspace{.25in}
			
		\begin{defn}[Function/Domain/Range]
			A \textbf{function} $f$ is a rule that assigns to each element $x$ in a set $D$ \textbf{one} element, called $f(x)$, in a set $E$.  The set $D$ is called the \textbf{domain} of the function.  The \textbf{range} of $f$ is the set of all possible values of $f(x)$ as $x$ varies throughout the domain. 
		\end{defn}	
		
		\begin{defn}[Independent Variable/Dependent Variable]
			A symbol that represents an arbitrary element in the domain is called an \showto{ins}{\fbox{independent variable}} \showto{st}{ \blank{1.3} \\ \\ \blank{2}}.  A symbol representing an arbitrary element in the range of a\showto{st}{\\[15pt]} function is called a \showto{ins}{\fbox{dependent variable}} \showto{st}{ \blank{3}}.
		\end{defn}
			\newpage
			
		\begin{ex}
			Let $C(t)$ represent the number of courses offered campus-wide during the week at time $t$, and $O(t)$ represent the number of students walking on the South Oval at time $t$ last Monday.   Is $C$ a function?  What about $O$?
		\end{ex}	
			\vs{1}
			
		\begin{ex}
			Fill out the table with the domain and range of the given function.  Write your answer in interval notation.
			\begin{center}
				\tabulinesep = 3mm
				\begin{tabu}to .95\textwidth {| X[c] | X[c] | X[c] |} \hline
					\textbf{Function}	&	\textbf{Domain}	& \textbf{Range} \\ \hline
					%						&					& \\ 
					$\sqrt{2+x}$			&					& \\ \hline
					%						&					& \\ 
					$\dfrac{x^2-1}{x-1}$	&					& \\ \hline
					%						&					& \\ 
					$x^3-6.2x^2+x-1$		&					& \\ \hline

				\end{tabu}
			\end{center}
		\end{ex}
			
		\begin{ex}
			Assuming $h\neq 0$, simplify the \emph{difference quotient} $\dfrac{f(1+h)-f(1)}{h}$ where $f(x) = 3x^2 -2x + 1$
		\end{ex}
			\vs{2}
			\newpage
	\subsubsection*{Representations of Functions}
	\addcontentsline{toc}{subsubsection}{Representations of Functions}
		In mathematics, particularly applied mathematics, we need to be able to interpret real-world phenomena in four ways: numerically, algebraically, verbally, and graphically.  

		\begin{ex}
			The price of gas at a certain 7-11 in Norman was \$4.37 per gallon on June 26th.  
			\begin{enumerate}[(a)]
				\item Is this information presented numerically, algebraically, verbally, or graphically?
					\vs{.5}
					
				\item Represent this situation in the other three ways.
					\vs{1}
			\end{enumerate}
		\end{ex}
				
		\begin{ex}
			A rectangular storage container has an open top, and a volume of 20 m$^3$.  The length of its base is twice its width.  Material for the base costs \$5 per square meter; material for the sides costs \$3 per square meter.  Express the cost of materials as a function of the width of the base.
		\end{ex}
			\vs{1}
			\newpage
			
		\begin{rmk}[Vertical Line Test]
			A curve in the $xy-$plane is the graph of a function of $x$ if and only if
			\showto{ins}{
				no vertical line intersects the curve more than once
			}
			\showto{st}{
			\\ \\ \\ \\
			}
		\end{rmk}
		\begin{ex}
			Are both of these graphs functions?  Why or why not?
			\begin{center}
				\begin{tabular}{lr}
					\begin{tikzpicture}
						\begin{axis}[
						axis x line = middle,
    						axis y line = middle,
	    					every axis y label/.style={at={(ticklabel cs:1.1)}},
							y label style={at={(axis description cs:0,1.1)},anchor=north},
	    					ylabel = {$y$},
    						every axis x label/.style= {at ={(ticklabel cs:1)}},
    						x label style={at={(axis description cs:1.1,.48)},anchor=east},
    						xlabel = {$x$},
						trig format plots = rad,
						xmin = 0, xmax = 6.5,
						ymin = -1.5, ymax = 1.5
						]
						\addplot[thick, smooth, domain = 0:2*pi] {cos(x)};
						\end{axis}
					\end{tikzpicture}
				&
					\begin{tikzpicture}
	  	  				\begin{axis}[
    						axis x line = middle,
    						axis y line = middle,
	    					every axis y label/.style={at={(ticklabel cs:1.1)}},
							y label style={at={(axis description cs:.5,1.1)},anchor=north},
	    					ylabel = {$y$},
    						every axis x label/.style= {at ={(ticklabel cs:1)}},
    						x label style={at={(axis description cs:1.1,.65)},anchor=east},
    						xlabel = {$x$},
							xmin=-4.5,xmax=4.5,
       			    		ymin=-9.5,ymax=4.5,
			       		    xtick = {-4,-2,2,4},
	    		   		    ytick = {-8,-6,-4,-2,2,4},
        		    	]
        			    \addplot [domain=-3:3,samples=50]({x^3-3*x},{3*x^2-9}); 
	   				 \end{axis}
					\end{tikzpicture}
				\end{tabular}
			\end{center}
		\end{ex}
			\vs{1}
			
		\begin{ex} 
			Below are numerical expressions for the relations $h$ and $k$.  Is $h$ a function?  What about $k$?
			\begin{center}
				{\renewcommand{\arraystretch}{1.2}
				\begin{tabular}{|c|c|c|c|c|c|c|} \hline
					$x$ & 0 & 1 & 1 & 2 & 5 & 6 \\ \hline
					$h(x)$ & 0& 1 & 2 & 3 & 4 & 5 \\ \hline
				\end{tabular}\hspace*{15pt}
				\begin{tabular}{|c|c|c|c|c|c|c|} \hline
						$t$ & 0 & 1 & 1 & 2 & 5 & 6 \\ \hline
						$k(t)$ & 0 & 1 & 1 & 3 & 4 & 5 \\ \hline
					\end{tabular}
				}
				\end{center}
		\end{ex}			
			\vs{1}
			\newpage
			
		\subsubsection*{Pre-Class Activities}
		\addcontentsline{toc}{subsubsection}{Pre-Class Activities}
		\begin{ex}
			If $f(x) = 6x^2 -3x + 1$, find the following: $f(1)$, $f(-2)$, $f(a)$, $f(-a)$, $f(a+1)$, $2f(a)$, $f(2a)$, $f(a^2)$, $[f(a)]^2$, and $f(a+h)$.
		\end{ex}	
			\vs{1}
			
		\begin{ex}
			Evaluate the difference quotient $\dfrac{f(x) - f(3)}{x-3}$ for $f(x) = x^3$.
		\end{ex}
			\vs{1}
			\newpage
			
		\begin{ex}
			Find the domain of the functions below:
			\begin{enumerate}[(a)]
				\item $f(x) = \dfrac{3x^4 - 5}{x^2 +2x - 8}$
					\vs{1}
					
				\item $g(k) = \sqrt[3]{1-7k}$
					\vs{1}
				\item $h(t) = \sqrt{2-t} - \sqrt{3+t}$	
					\vs{1}
			\end{enumerate}
		\end{ex}
		
		\begin{ex}
			Without referring to the vertical line test, explain why the graph below is \textbf{not} a function.
			\begin{flushleft}
				\begin{tikzpicture}
	  	  			\begin{axis}[
    						axis x line = middle,
    						axis y line = middle,
	    					every axis y label/.style={at={(ticklabel cs:1.1)}},
							y label style={at={(axis description cs:.5,1.1)},anchor=north},
	    					ylabel = {$y$},
    						every axis x label/.style= {at ={(ticklabel cs:1)}},
    						x label style={at={(axis description cs:1.1,.65)},anchor=east},
    						xlabel = {$x$},
							xmin=-4.5,xmax=4.5,
       			    		ymin=-9.5,ymax=4.5,
			       		    xtick = {-4,-2,2,4},
	    		   		    ytick = {-8,-6,-4,-2,2,4},
        		    	]
        			    \addplot [domain=-3:3,samples=50]({x^2-3},{x^3-3*x}); 
	   				 \end{axis}
				\end{tikzpicture}
			\end{flushleft}
		\end{ex}
		\newpage
		
	\subsection*{In-Class}
	\addcontentsline{toc}{subsection}{In-Class}
	\subsubsection*{Piecewise Defined Functions}
	\addcontentsline{toc}{subsubsection}{Piecewise Defined Functions}
		\begin{defn}[Piecewise Function]
			A \textbf{piecewise function} is a function defined by different formulas in different parts of their domains.
		\end{defn}
						
		\begin{ex}
			A quick example of a piecewise function is the \emph{absolute value function}: 
				\[f(x) = |x| = \begin{cases}-x & x < 0 \\x & x \geq 0  \end{cases}\]
				
			\begin{enumerate}[(a)]
				\item What is $f(-5)?$  What about $f(1)$?
					\vs{.5}
					
				\item What is $f(0)$?  Why?
					\vs{.5}
					
				\item Sketch $|x|$ on the interval $-5\leq x \leq 5$.
					\vs{1}
			\end{enumerate}
		\end{ex}
		
		
		\begin{ex}
			A function $h$ is defined by $h(x) = \begin{cases}3-x & x< 2\\ x^2+x & x \geq 2 \end{cases}$
			\begin{enumerate}[(a)]
				\item Evaluate $h(-2)$, $h(3)$, and $h(2)$.
					\vs{1}

					
				\item Sketch the graph of $h$
					\vs{1}
					
			\end{enumerate}
		\end{ex}
			\newpage
			
		\begin{ex}
			Write the absolute value function $f(x) = |2x-3|$ as a piecewise function
		\end{ex}
			\vs{1}
			
	\subsubsection*{Symmetry}
	\addcontentsline{toc}{subsubsection}{Symmetry}
		\begin{defn}[Even/Odd Function]
			A function $f$ is said to be \textbf{even} if it has the property that\\
			\showto{ins}{
				\[f(-x) = f(x)\]
			}
			\showto{st}{
				\\ \\ \\
			}
			A function $g$ is said to be \textbf{odd} if it has the property that\\
			\showto{ins}{
				\[g(-x) = -g(x)\]
			}
			\showto{st}{
				\\ \\ \\
			}
		\end{defn}
			
		\begin{ex}
			Determine if the following functions are even, odd, or neither.
			\begin{enumerate}[(a)]
				\item $f(x) = x^7 + x^5 - x$
					\vs{1}
					
				\item $g(x) = 3-x^2$
					\vs{1}
					
				\item $h(x) = x^2 - x^3$
					\vs{1}
					
			\end{enumerate}
		\end{ex}
			\newpage
			
	\subsubsection*{Increasing and Decreasing Functions}
	\addcontentsline{toc}{subsubsection}{Increasing and Decreasing Functions}
		\begin{defn}[Increasing/Decreasing]
			Let $f$ be a function defined on some input interval.  $f$ is said to be
			\showto{ins}{
				\begin{itemize}
					\item \textbf{increasing} if \fbox{the output values increase on the interval}
					\item \textbf{decreasing} if \fbox{the output values decrease on the interval}
				\end{itemize}
			}
			\showto{st}{\\ \\
				\begin{itemize}
					\setlength\itemsep{15pt}
					\item \textbf{increasing} if \blank{4.5}
					\item \textbf{decreasing} if \blank{4.5}
				\end{itemize}
			}
		\end{defn}
		
		\begin{ex}
			Identify the intervals for which the function is increasing and decreasing.
			\begin{flushleft}
				\begin{tikzpicture}
					\begin{axis}[
					axis x line = middle,
				    	axis y line = middle,
				    	every axis y label/.style={at={(ticklabel cs:1.1)}},
					y label style={at={(axis description cs:.4,1.1)},anchor=north},
				    	ylabel = {$y$},
				    	every axis x label/.style= {at ={(ticklabel cs:1)}},
				    	x label style={at={(axis description cs:1.1,.6)},anchor=east},
				    	xlabel = {$x$},
					xmin=-3.5,xmax=5,
					xtick = {-3,-2,-1,1,2,3,4},
					]
					\addplot[thick,smooth,domain = -3:4] {(1/3)*x^3 -(1/2)*x^2 -2*x};
					\end{axis}
				\end{tikzpicture}
			\end{flushleft}
		\end{ex}
		\newsec
		
		\subsection*{After Class}
		\addcontentsline{toc}{subsection}{After Class}
		\begin{ex}
			Evaluate $h(3), h(0),$ and $h(2)$ for the function $h(x) = \begin{cases}3 - \dfrac{1}{2}x & x < 2\\ 2x - 5 & x\geq 2 \end{cases}$.  Then, sketch the graph of $h(x)$.
		\end{ex}
			\newpage
			
		

		\begin{ex}
			Sketch the graph of $f(x) = x + |x|$ and $g(x) = \begin{cases} |x| & |x| \leq 1\\ 1 & x > 1\end{cases}$.
		\end{ex}
			\vs{1}
			
		\begin{ex}
			If the point $(3,5)$ is on the graph of an even function, what other point must also be on the graph?  What about if the function is odd?  Justify your answers.
		\end{ex}
			\vs{1}
			
		\begin{ex}
			Is the function $f(t) = t|t|$ even, odd, or neither?  Explain.
		\end{ex}
			\vs{1}
		\clearpage
\end{document}