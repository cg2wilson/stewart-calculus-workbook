\documentclass[notes]{subfiles}
\begin{document}
	\addcontentsline{toc}{section}{11.3 - The Integral Test \& Estimates of Sums}
	\refstepcounter{section}
	\fancyhead[RO,LE]{\bfseries \nameref{cs113}} 
	\fancyhead[LO,RE]{\bfseries \small \currentname}
	\fancyfoot[C]{{}}
	\fancyfoot[RO,LE]{\large \thepage}	%Footer on Right \thepage is pagenumber
	\fancyfoot[LO,RE]{\large Chapter 11.3}
	
\section*{The Integral Test \& Estimates of Sums}\label{cs113}
	\subsection*{Before Class}
	\subsubsection*{The Integral Test}
		\begin{ex}
			Consider the sequence $a_n = \dfrac{1}{n^2}$.
			\begin{enumerate}[(a)]
				\item Find the first five terms of the sequence.
					\vs{1}
				\item Use rectangles to represent the values of the first five terms.  Using the geometric information, how can we find $s_5$?
					\vs{1}
				\item Now, graph the function $f(x) = \dfrac{1}{x^2}$ and copy the rectangles you drew above.  What can you say about the relationship between the value of the series and the value of the integral, as the inputs increase?
					\vs{2}
			\end{enumerate}
		\end{ex}
			\newpage
			
		\begin{rmk}[The Integral Test]
			Suppose $f$ is a continuous, positive, \emph{eventually} decreasing function on the interval $[k,\infty)$.  Let $a_n = f(n)$.  Then,\\
			\begin{itemize}
				\setlength \itemsep{20pt}
				
				\item If $\ds \int_k^\infty f(x)\, dx$ is convergent, then \blank{3}.
				\item If $\ds \int_k^\infty f(x)\, dx$ is divergent, then \blank{3}.
			\end{itemize}
		\end{rmk}
		
		\begin{ex}
			Show that the series $\ds \sum_{n = 1}^\infty \dfrac{1}{n^2}$ converges.  Can we determine the value of the series from the Integral Test?
		\end{ex}
			\vs{1}
			
		\begin{ex}
			Show that the series $\ds \sum_{n =1 }^\infty \dfrac{1}{n}$ diverges.  This is an important series, called the \emph{harmonic series}.
		\end{ex}
			\vs{1}
			\newpage
			
	\subsection*{Pre-Class Activities}
		\begin{ex}
			Use this space to write any questions you might have from the videos.
		\end{ex}	
			\vs{.5}
			
		\begin{ex}
			Use the Integral Test to determine if the series $\ds \sum_{n = 1}^\infty \dfrac{1}{n^2 + 9}$ converges or diverges.  Be sure to check the hypotheses of the Integral Test.
		\end{ex}
			\vs{1}
			
		\begin{ex}
			Show that the series $\ds \sum_{n = 1}^\infty \dfrac{\ln n}{n}$ diverges using the Integral Test.  Be sure to verify each hypothesis of the Integral Test.
		\end{ex}
			\vs{1}
			\newpage
			
	\subsection*{In Class}
		
		\begin{ex}
			A series of the form $\ds \sum_{n = 1}^\infty \dfrac{1}{n^p}$ is called a \emph{p-series}.  Determine the values of $p$ for which a $p-$series will converge.
		\end{ex}
			\vs{1}
			
		\begin{rmk}[$p-$series Test]
			A series of the form $\ds \sum_{n = 1}^\infty \dfrac{1}{n^p}$ converges if \blank{1.2}, and diverges if \blank{1.2}.
		\end{rmk}
		
		\begin{ex}
			Determine convergence or divergence of the series $\ds \sum_{n = 2}^\infty \dfrac{1}{n^2(n^2-1)}$.
		\end{ex}
			\vs{1}
			\newpage
	
	\subsubsection*{Estimating the Sum of a Series}	
		\begin{defn}[Remainder of a Sum]
			For the series $\ds \sum a_n$, the \textbf{remainder} $R_n$is defined to be the value\\
				\[R_n =\hspace{2in}\]
				\\
			where $s$ is \blank{2} and $s_n$ is \blank{3.2}.  $R_n$ is the error made when $s_n$ is used as an approximation to the total sum.
		\end{defn}
		
		\begin{rmk}[Remainder Estimate for the Integral Test]
			Suppose $f(k) = a_k$, where $f$ is a continuous, positive, decreasing function for $x\geq n$, and that\\[15pt] $\ds \sum a_n$ is convergent.  If $R_n =$ \blank{2}, then\\
				\[\leq R_n \leq\]
				\\
			Rearranging, we get another expression:\\
				\[\leq s \leq\]
		\end{rmk}
		
		\begin{ex}
			$ $
			\begin{enumerate}[(a)]
				\item Approximate the sum of the series $\ds \sum \dfrac{1}{n^3}$ by using the sum of the first 10 terms.  Estimate the error involved in this approximation.
					\vs{1}
					\newpage
					
				\item How many terms are required to ensure that the sum is accurate to within 0.0005?
					\vs{1}
					
			\end{enumerate}
		\end{ex}
		
		\begin{ex}
			Use $n = 10$ to estimate the sum of the series $\ds \int_{n =1 }^\infty \dfrac{1}{n^3}$
		\end{ex}
			\vs{1}
			
		\begin{ex}
			Find the sum of the series $\ds \sum ne^{-2n}$ correct to four decimal places.
		\end{ex}
			\vs{1}
			\newpage
			
	\subsubsection*{Examples}
		\begin{ex}
			Determine if the series $\ds \sum_{n = 1}^\infty \dfrac{1}{n^{\sqrt{2}}}$ converges or diverges.
		\end{ex}
			\vs{1}
			
		\begin{ex}
			Does $\ds \sum_{n = 1}^\infty \dfrac{\sqrt{n} + 4}{n^2}$ converge or diverge?
		\end{ex}
			\vs{1}
			
		\begin{ex}
			Does $\ds \sum_{k = 1}^\infty ke^{-k}$ converge or diverge?
		\end{ex}
			\vs{1}
			\newpage
			
		\begin{ex}
			Find the values of $p$ for which the series $\ds \sum_{n = 2}^\infty \dfrac{1}{n(\ln n)^p}$ converges.
		\end{ex}
			\vs{1}
			
		\begin{ex}
			Estimate $\ds \sum_{n = 1}^\infty (2n + 1)^{-6}$ correct to five decimal places.
		\end{ex}
			\vs{1}
			\newpage
			
			
	\subsection*{After Class Activities}
		\begin{ex}
			Determine if $\ds \sum_{n =1 }^\infty \dfrac{1}{n^2 + n^3}$ converges or diverges.
		\end{ex}
			\vs{1}
			
		\begin{ex}
			Does $1 + \dfrac{1}{2\sqrt{2}} + \dfrac{1}{3\sqrt{3}} + \dfrac{1}{4\sqrt{4}} + \dfrac{1}{5\sqrt{5}}+ \cdots$ converge or diverge?
		\end{ex}
			\vs{.5}
			
		\begin{ex}
			The series $\ds \sum_{n = 1}^\infty \dfrac{\cos \pi n}{\sqrt{n}}$ converges, but we can't use the Integral Test to prove that.  Why not?
		\end{ex}
			\vs{.5}
			
		\begin{ex}
			How many terms of the series $\ds \sum-{n =2}^\infty \dfrac{1}{n(\ln n)^2}$ are needed to find its sum to within 0.01?
		\end{ex}
			\vs{1}
\clearpage
\end{document}