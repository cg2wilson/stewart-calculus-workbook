\documentclass[notes]{subfiles}

\begin{document}
	\addcontentsline{toc}{section}{5.3 - Volumes by Cylindrical Shells}
	\refstepcounter{section}
	\fancyhead[RO,LE]{\bfseries \large\nameref{cs53}} 
	\fancyhead[LO,RE]{\bfseries \currentname}
	\fancyfoot[C]{{}}
	\fancyfoot[RO,LE]{\large \thepage}	%Footer on Right \thepage is pagenumber
	\fancyfoot[LO,RE]{\large Chapter 5.3}
	
\section*{Volumes by Cylindrical Shells}\label{cs53}
	\subsection*{Before Class}
	\addcontentsline{toc}{subsection}{Before Class}
	\subsubsection*{Slicing With Cylinders}
	\addcontentsline{toc}{subsubsection}{Slicing With Cylinders}
		\begin{ex}
			Let $f(x) = x-x^2$.
			\begin{enumerate}[(a)]
				\item Sketch the region bounded by $f(x)$ and the $x-$axis.
					\vs{2}
				\item Set up (but do not solve) an integral to find the volume of the solid created by rotating the region about the line $y = 0$.  Why can we use the disk method?
					\vs{1}
				\item Are there problems if we instead rotate about the line $x = 0$?  Explain with pictures, words, etc.
					\vs{1}
			\end{enumerate}
		\end{ex}

			\newpage
			
		\begin{ex}
			Again consider $f(x) = x-x^2$.
			\begin{enumerate}[(a)]
				\item Draw four midpoint rectangles for the area of the region bounded by $f(x)$ and the $x-$axis.  Be sure to label things appropriately!
					\vs{2}
					
				\item Sketch what happens to the rectangles when rotated about the line $x = 0$.
					\vs{1.5}
					
				\item How can we find the surface area of the shape from part (b)?  
					\vs{1}
					\newpage
					
				\item Use your answers from parts (b) and (c) to approximate the volume of the solid.
					\vs{2}
					
				\item Write an expression that uses ``infinitely many'' rectangles to approximate the the volume of the solid.
					\vs{1}
					
				\item Convert your answer from part (d) into an integral expression, and evaluate it.
					\vs{2}
			\end{enumerate}
		\end{ex}
			\newpage
	
	\subsubsection*{The Shell Method}
	\addcontentsline{toc}{subsubsection}{The Shell Method}
		\begin{rmk}[Method of Cylindrical Shells]
			Let $f(x)$ be a continuous function on the interval $[a,b]$.  Then, the volume of the solid created by rotating the region bounded by $f(x)$ and the $x-$axis about the line $x = 0$ is given by
			\showto{ins}{
				\[\int_a^b 2\pi r(x) f(x)\, dx\]
			}
			\showto{st}{
				\vspace*{1in} \\
			}
			where $r(x)$ is the
			\showto{ins}{
				\fbox{radius function}.
			}
			\showto{st}{
				\blank{3.5}.
			}
		\end{rmk}
		
		\begin{ex}
			Consider the region bounded by the curve $y = 2\sqrt{x}$, the $x-$axis, and the line $x = 4$.  
			\begin{enumerate}[(a)]
				\item Sketch and label the region.
					\vs{1}
					
				\item Sketch and label a typical approximating cylinder, when rotated about $x = 0$.
					\vs{1}
					
				\item Use the formula for the cylinder to set up an integral for the volume of the resulting solid, then evaluate it.
					\vs{1}
			\end{enumerate}
		\end{ex}
			\newpage
			
	\subsubsection*{Pre-Class Activities}
	\addcontentsline{toc}{subsubsection}{Pre-Class Activities}
		\begin{ex}
			If you were presented with a problem, how would you know whether to use the disk method, washer method, or shell method?  Examples 1 and 2 might be good places to get ideas.
		\end{ex}
			\vs{1}
			
		\begin{ex}
			Find the volume of the solid obtained by rotating the region bouded by $y = 2x^2 - x^3$ and $y = 0$ about the $y-$axis.
		\end{ex}
			\vs{1} $ $\\ 
			\newsec
	
	\subsection*{In-Class}
	\addcontentsline{toc}{subsection}{In-Class}
		\begin{ex}
			Use the shell method to find the volume of the solid generated by revolving the region bounded by the curves $xy = 1$, $x = 0$, $y = 1$, and $y = 3$ about the $x-$axis.  Hint: Sketch the solid and its approximating cylinders.
		\end{ex}
			\vs{1}
			\newpage
			
			
		\begin{ex}
			Find the volume of the solid created by rotating the region bounded by the curves $y = x^3$, $y = 0$, $x = 1$, and $x = 2$ about the $y-$axis.  Be sure to label your diagram, if you choose to draw one.
		\end{ex}
			\vs{1}

			
		\begin{ex}
			Find the volume of the solid created by rotating the first-quadrant region bounded by the curves $y = 4x-x^2$, and $y = x$ about the $y-$axis.  Be sure to label your diagram, if you choose to draw one.
		\end{ex}
			\vs{1}
			\newpage
			
		\begin{ex}
			Find the volume of the solid created by rotating the region bounded by the curves $x = 2y^2$, $y\geq 0$, and $x=8$ about $y = 2$.  Hint: Draw and label a diagram.  Remember that this is a function of $y$, not a function of $x$!
		\end{ex}
			\vs{1}

			
		\begin{ex}
			Find the volume of the solid created by rotating the region bounded by the curves $y = x^3$, $y = 8$, $x = 0$ about $x = 2$.  Be sure to label your diagram, if you choose to draw one.
		\end{ex}
			\vs{1} 
			\newpage
			
	\subsection*{After Class}
	\addcontentsline{toc}{subsection}{After Class}
		
		\begin{ex}
			Find the volume of the solid created by rotating the region bounded by the curves $x = 2y^2$, $x = y^2 + 1$ about $y = -2$.
		\end{ex}
			\vs{1}
			
		\begin{ex}
			Sketch the region described in the volume integral below:
				\[V = \int_1^4 \dfrac{2\pi (y+2)}{y^2}\, dy\]
		\end{ex}
			\vs{1}
			\newpage

		\begin{ex}
			The region bounded by the curves $y = -x^2 + 6x-8$ and $y = 0$ is rotated about the $y-$axis.
				\begin{enumerate}[(a)]
					\item Sketch the region described above.
						\vs{.5}
						
					\item Of our three methods, which one do you want to use?  Give a quick explanation why you chose that particular method.
						\vs{.5}
						
					\item Compute the volume of the solid.
						\vs{1}
				\end{enumerate}
		\end{ex}
			\newpage
			
		\begin{ex}	
			Consider the region bounded by the curves $y = 4x$, $y = 0$, and $x = 2$.  Find the volume of the solid formed when the region is rotated about the given lines:
			\begin{enumerate}[(a)]
				\item The $y-$axis
					\vs{1}
					
				\item $x = 4$
					\vs{1}
			\end{enumerate}
		\end{ex}
				
	\clearpage
\end{document}