\documentclass[notes]{subfiles}
\begin{document}
	\addcontentsline{toc}{section}{6.3 - Logarithmic Functions}
	\refstepcounter{section}
	\fancyhead[RO,LE]{\bfseries \nameref{cs63}} 
	\fancyhead[LO,RE]{\bfseries \small \currentname}
	\fancyfoot[C]{{}}
	\fancyfoot[RO,LE]{\large \thepage}	%Footer on Right \thepage is pagenumber
	\fancyfoot[LO,RE]{\large Chapter 6.3}
	
\section*{Logarithmic Functions}\label{cs63}
	\subsection*{Before Class}
	\addcontentsline{toc}{subsection}{Before Class}
	\subsubsection*{Logarithmic Functions}
	\addcontentsline{toc}{subsubsection}{Logarithmic Functions}
		\showto{st}{
		\begin{defn}[Logarithmic Function]
			Let $y = b^x$.  The inverse of this exponential function, called a \textbf{logarithmic function}, is\\[20pt] defined using the relationship \blank{3}.
		\end{defn}
		}
		
		\showto{ins}{
		\begin{defn}[Logarithmic Function]
			Let $y = b^x$.  The inverse of this exponential function, called a \textbf{logarithmic function}, is defined using the relationship $y = b^x \iff x = \log_b(y)$.
		}
		
		\begin{ex}
			Find the following:
			\begin{enumerate}[(a)]
				\item $\log_4(16)$
					\vs{.5}
					
				\item $\log_{10} (0.01)$
					\vs{.5}
					
				\item $\log_3 (243)$
					\vs{.5}
					
				\item $\log_5 (-5)$
					\vs{.5}
			\end{enumerate}
		\end{ex}
		
		\showto{st}{
		\begin{rmk}[Cancellation Properties]
			\begin{itemize}
				\setlength\itemsep{30pt}
				\item $\log_b (b^x) =$ \blank{3}
				\item $b^{\log_b(x)} = $ \blank{3}
			\end{itemize}
		\end{rmk}
		}
		
		\showto{ins}{
		\begin{rmk}[Cancellation Properties]
			\begin{itemize}
				\item $\log_b (b^x) = x$ for $x\in \R$ 
				\item $b^{\log_b(x)} = x$ for $x > 0$
			\end{itemize}
		\end{rmk}
		}
			\newpage
		
		\showto{st}{
		\begin{rmk}[Properties of Logarithmic Functions]
			Let $f(x) = \log_b(x)$.  Then, $f(x)$ has the following properties:\\ \\
			\begin{itemize}
				\setlength\itemsep{25pt}
				\item Domain: \blank{1.75} 
				\item Range: \blank{1.75}
				\item If \blank{2}, then $f(x)$ is increasing
				\item If \blank{2}, then $f(x)$ is decreasing
				\item $\ds\lim_{x\to \infty} f(x) = \begin{cases}\blank{.75}& \text{if } \blank{1.5}\\ & \\ \blank{.75}& \text{if } \blank{1.5} \end{cases}$
				\item $\ds\lim_{x\to 0^+} f(x) = \begin{cases}\blank{.75}& \text{if } \blank{1.5}\\ & \\ \blank{.75}& \text{if } \blank{1.5} \end{cases}$
			\end{itemize}
		\end{rmk}
		}
		
		\showto{ins}{
		\begin{rmk}[Properties of Logarithmic Functions]
			Let $f(x) = \log_b(x)$.  Then, $f(x)$ has the following properties:\\ \\
			\begin{itemize}
				\setlength\itemsep{10pt}
				\item Domain: $(0,\infty)$ 
				\item Range: $(-\infty,\infty)$
				\item If $b > 1$, then $f(x)$ is increasing
				\item If $0 < b< 1$, then $f(x)$ is decreasing
				\item $\ds\lim_{x\to \infty} f(x) = \begin{cases}\infty& \text{if } $b > 1$\\ & \\ -\infty& \text{if } $0<b<1$ \end{cases}$
				\item $\ds\lim_{x\to 0^+} f(x) = \begin{cases}-\infty& \text{if } $b > 1$\\ & \\ \infty & \text{if } $0<b<1$ \end{cases}$
			\end{itemize}
		\end{rmk}
		}
			\vs{1}
			
		\showto{st}{
		\begin{rmk}[Logarithm Rules]
			For $x,y>0$ and $r\in \R$, the following properties hold:\\ \\
			\begin{itemize}
				\setlength\itemsep{15pt}
				\item $\log_b(xy) = $\blank{2.5}
				\item $\log_b\lrpar{\dfrac{x}{y}} = $\blank{2.5}
				\item $\log_b(x^r) = $\blank{2.5}
			\end{itemize}
		\end{rmk}
		}
		\showto{ins}{
		\begin{rmk}[Logarithm Rules]
			For $x,y>0$ and $r\in \R$, the following properties hold:\\ \\
			\begin{itemize}
				\item $\log_b(xy) = \log_b(x) + \log_b(y)$
				\item $\log_b\lrpar{\dfrac{x}{y}} = \log_b(x) - \log_b(y)$
				\item $\log_b(x^r) = r\log_b(x)$
			\end{itemize}
		\end{rmk}
		}
			\newpage
			
		\begin{ex}
			Use the rules of logarithms to write the following as a single logarithm:
			\begin{enumerate}[(a)]
				\item $2\log_3(x) + 3\log_3(y) - \log_3(z)$
					\vs{1}
					
				\item $\log_2(160) + \log_2(10)$
					\vs{1}
			\end{enumerate}
		\end{ex}
		
		\begin{ex}
			Use the rules of logarithms to expand the given quantity:
			\begin{enumerate}[(a)]
				\item $\log_9(\sqrt[3]{ab})$
					\vs{1}
					
				\item $\log_{10}\lrpar{\lrpar{\dfrac{x+1}{x-2}}^2}$
					\vs{1}
			\end{enumerate}
		\end{ex}
			\newpage
			
	\subsection*{Pre-Class Activities}	
	\addcontentsline{toc}{subsection}{Pre-Class Activities}	
		\begin{ex}
			Find the exact value of the expression:
			\begin{enumerate}[(a)]
				\item $\log_2(32)$
					\vs{.5}
					
				\item $\log_{1.5}(2.25)$
					\vs{.5}
					
				\item $\log_8 (60) - \log_8 (3) - \log_8 (5)$
					\vs{.5}
					
			\end{enumerate}
		\end{ex}	
		
		\begin{ex}
			Write $\log_{10}(4) + \log_{10}(a) - \dfrac{1}{3}\log_{10} (a+1)$ as a single logarithm.
		\end{ex}
			\vs{.5}
			
		\begin{ex}
			Can $\log_b(x) + \log_c(y)$ be written as a single logarithm? Why or why not?
		\end{ex}
			\vs{.5}
			
		\begin{ex}
			Let $f(x) = \log_5(8x-x^4)$.  Use log rules to completely simplify $f(x)$, then use limit laws to compute $\ds \lim_{x\to 2^-} f(x)$.
		\end{ex}
			\vs{1}
			
		\begin{question}
			Use this space to write any questions or thoughts you have from the videos.
		\end{question}
			\vs{.5}
			\newpage
			
	\subsection*{In Class}
	\addcontentsline{toc}{subsection}{In Class}
	\subsubsection*{The Natural Logarithm}
	\addcontentsline{toc}{subsubsection}{The Natural Logarithm}
		\begin{defn}[The Natural Logarithm]
			The \textbf{natural logarithm} is the logarithm with base $e$.  $\log_e(x)$ is written as $\ln x$.
		\end{defn}
		
		\begin{ex}
			Find $x$, if $e^x = 5$
		\end{ex}
			\vs{.5}
			
		\begin{ex}
			Sketch the graph of $y = \ln(x-1) + 2$
		\end{ex}
			\vs{1}
			
		\begin{ex}
			Solve the equation $e^{3-5x} = 10$
		\end{ex}
			\vs{1}
			\newpage
		
		\begin{ex}
			Solve the equations for $x$:
			\begin{enumerate}[(a)]
				\item $e^{7-4x} = 6$
					\vs{1}
					
				\item $\ln(3x-10) = 2$
					\vs{1}
					
				\item $\ln(x^2-1) = 3$
					\vs{1}
					
				\item $\log_2(mx) = c$
					\vs{1}
					
				\item $e-e^{-2x} = 1$
					\vs{1}
					
				\item $10(1+e^{-x})^{-1} = 3$
					\vs{1}
					
				\item $e^{2x} - e^x - 6 = 0$
					\vs{1}
			\end{enumerate}
		\end{ex}	
			\newpage
		
		\begin{ex}
			Find the following limits:
			\begin{enumerate}[(a)]
				\item $\ds\lim_{x\to 3^+} \ln(x^2-9)$
					\vs{1}
					
				\item $\ds\lim_{x\to 2^-} \log_5(8x-x^4)$
					\vs{1}
					
				\item $\ds\lim_{x\to 0^+} \ln (\sin x)$
					\vs{1}
			\end{enumerate}
		\end{ex}	
		
		\begin{ex}
			If $f(x) = \sqrt{3-e^{2x}}$, find the domain of $f$, the inverse function $\inv{f}$, and the domain of $\inv{f}$.
		\end{ex}
			\vs{2}
			\newpage
			
		\begin{ex}
			Find the inverse of the function $g(x) = \log_4(x^3 + 2)$.
		\end{ex}	
			\vs{1}
			
		\begin{ex}
			Where is the function $f(x) = e^{3x} - e^x$ increasing?
		\end{ex}
			\vs{1}
			
		\begin{ex}
			Find an equation of the tangent to the curve $y = e^{-x}$ that is perpendicular to the line $2x-y = 8$.
		\end{ex}
			\vs{1}
			\newpage
			
		\showto{st}{
		\begin{rmk}[Change of Base Formula]
			For any positive number $b$ ($b\neq 1$), we have
				\[\]
		\end{rmk}
		}
		\showto{ins}{
		\begin{rmk}[Change of Base Formula]
			For any positive number $b$ ($b\neq 1$), we have
				\[\log_b x = \dfrac{\log_c x}{\log_c b}\]
		\end{rmk}
		}
		
		\begin{ex}
			Write the logarithm $\log_3(7)$ in terms of the natural logarithm.
		\end{ex}
			\vs{1}
			
		\begin{ex}
			Use the change of base formula to write $\dfrac{1}{\log_8 6}$ as a single logarithm.
		\end{ex}
			\vs{1} $ $
			\newsec
			
	\subsection*{After Class Activities}
	\addcontentsline{toc}{subsection}{After Class Activities}
		\begin{ex}
			Solve the equation $\ln x + \ln (x-1) = 1$
		\end{ex}
			\vs{1}
			\newpage
			
		\begin{ex}
			Find the limits:
			\begin{enumerate}[(a)]
				\item $\ds \lim_{x\to \infty}\lrpar{\ln(1+x^2) - \ln(1+x)}$
					\vs{1}
					
				\item $\ds \lim_{x\to \infty}\lrpar{\ln(2+x) - \ln(1+x)}$
					\vs{1}
			\end{enumerate}
		\end{ex}
		
		\begin{ex}
			Find the domain of the function $\log_2(x^2 + 3x)$
		\end{ex}
			\vs{1}
			\newpage
			
		\begin{ex}
			Find the domain of the function $f(x) = \ln (2+\ln x)$, the inverse function $\inv{f}$, and the domain of $\inv{f}$.
		\end{ex}
			\vs{1}
			
		\begin{ex}
			If $f(x) = 3^{2x-4}$, find $\inv{f}$.
		\end{ex}
			\vs{1}
			
		\begin{ex}
			On what interval is the curve $y = 2e^x - e^{-3x}$ concave down?
		\end{ex}
			\vs{1}
				
\clearpage
\end{document}
