\documentclass[notes]{subfiles}
\begin{document}
	\addcontentsline{toc}{section}{7.4 - Integration of Rational Functions by Partial Fractions}
	\refstepcounter{section}
	\fancyhead[RO,LE]{\bfseries \nameref{cs74}} 
	\fancyhead[LO,RE]{\bfseries \small \currentname}
	\fancyfoot[C]{{}}
	\fancyfoot[RO,LE]{\large \thepage}	%Footer on Right \thepage is pagenumber
	\fancyfoot[LO,RE]{\large Chapter 7.4}
	
\section*{Integration Using Partial Fractions}\label{cs74}
	\subsection*{Before Class}
	\addcontentsline{toc}{subsection}{Before Class}
	\subsubsection*{Distinct Linear Factors}
	\addcontentsline{toc}{subsubsection}{Distinct Linear Factors}
		\begin{ex}
			Find $\ds \int\dfrac{1}{x^2-6x-7}\, dx$.
			\begin{enumerate}[(a)]
				\item Why can we not use any of our previous methods on this problem?
					\vs{1}
				\item Before we can solve this, think about adding the fractions $\dfrac{1}{2} + \dfrac{1}{3}$.  What do you have to do in order to add these two fractions?
					\vs{.5}
				\item How could you \emph{decompose} the answer from above into two fractions?  Think about part (b)...
					\vs{1}
				\item Use your thought process in part (c) to decompose $f(x)$ into two fractions. 
					\vs{2}
					\newpage
					
				\item Integrate the resulting decomposition.
					\vs{1}
			\end{enumerate}
		\end{ex}
		
		The previous example gives us a process for integrating particular types of rational functions:
		\begin{rmk}[Functions With Two Distinct Linear Factors]
			Let $f(x) = \dfrac{P(x)}{Q(x)}$, where $\text{deg}(P) < \text{deg}(Q)$.  If $Q(x)$ can be written as $Q(x) = (x-a)(x-b)$, then 
				\[\dfrac{P(x)}{Q(x)} = \qquad \qquad\qquad\]
		\end{rmk}
		
		\begin{ex}
			Integrate $\ds \int \dfrac{x^3 + x}{x-1}\, dx$
		\end{ex}
			\vs{1.5}
			\newpage
		
		\begin{ex}
			Integrate $\ds \int \dfrac{1}{x^2-a^2}\, dx$ using the method of partial fractions.
		\end{ex}	
			\vs{1}
			
		\begin{ex}
			Integrate $\ds \int \dfrac{x^2+2x-1}{2x^3+3x^2-2x}\, dx$
		\end{ex}
			\vs{1}
			\newpage
			
		The previous example allows us to generalize our boxed comment from earlier:
		\begin{rmk}[Functions With $n$ Distinct Linear Factors]
			Let $f(x) = \dfrac{P(x)}{Q(x)}$, where $\text{deg}(P) < \text{deg}(Q)$.  If $Q(x)$ can be written as 
				\[Q(x) = (x-x_1)(x-x_2)\cdots(x-x_n)\]
			where each $x_i$ is distinct, then 
				\[\dfrac{P(x)}{Q(x)} = \qquad \qquad\qquad\]
		\end{rmk}
		\newsec
	\subsection*{Pre-Class Activities}
	\addcontentsline{toc}{subsection}{Pre-Class Activities}
		\begin{ex}
			Write any questions you have from the videos in this space.
		\end{ex}
			\vs{.5}
		\begin{ex}
			Find the general antiderivative of the function $f(x) = \dfrac{x^4}{x-1}$.
		\end{ex}
			\vs{1}
			\newpage
		
		\begin{ex}
			Integrate $\ds \int \dfrac{ax}{x^2-bx}\, dx$
		\end{ex}
			\vs{.5}
			
		\begin{ex}
			Integrate $\ds \int \dfrac{x}{(x-1)(x+1)(2x+1)}$
		\end{ex}
			\vs{1}
			\newpage
			
	\subsection*{In Class}
	\addcontentsline{toc}{subsection}{In Class}
	\subsubsection*{Repeated Linear Factors}
	\addcontentsline{toc}{subsubsection}{Repeated Linear Factors}
		\begin{ex}
			Find $\ds \int \dfrac{x^6}{x^2-4}\, dx$
		\end{ex}
			\vs{1}
			
		\begin{rmk}[Functions with Repeated Linear Factors]
			Let $f(x) = \dfrac{P(x)}{Q(x)}$, where $\text{deg}(P) < \text{deg}(Q)$.  If $Q$ can be written as
				\[Q(x) =(x-x_1)^{m_1}(x-x_2)^{m_2}\cdots (x-x_n)^{m_n}\]
			where each $x_i$ is distinct, then
				\[\dfrac{P(x)}{Q(x)} = \hspace{4in}\]
		\end{rmk}
		\begin{ex}
			Evaluate $\ds \int \dfrac{x}{(x-1)^2}\, dx$
		\end{ex}
			\vs{1}
			\newpage
			
		\begin{ex}
			Find $\ds \int \dfrac{x^4-2x^2+4x+1}{x^3-x^2-x+1}\, dx$
		\end{ex}
			\vs{1}
			
		\begin{ex}
			Evaluate $\ds \int \dfrac{x^3+4x^2+x-1}{x^3+x^2}\, dx$
		\end{ex}
			\vs{1}
			\newpage
			
	\subsubsection*{Irreducible Quadratic Factors}
	\addcontentsline{toc}{subsubsection}{Irreducible Quadratic Factors}
		\begin{defn}[Irreducible Quadratic]
			The polynomial $ax^2 + bx + c$ is said to be \textbf{irreducible} if \blank{2}.
		\end{defn}
		\begin{rmk}[Functions with An Irreducible Quadratic Factor]
			Let $f(x) = \dfrac{P(x)}{Q(x)}$, where $\text{deg}(P) < \text{deg}(Q)$.  If $Q(x)$ has an irreducible quadratic factor, then the decomposition of $f(x) = \dfrac{P(x)}{Q(x)}$ will have a term of the form
			\\ \\ \\
		\end{rmk}
		\begin{ex}
			Evaluate $\ds \int \dfrac{2x^2-x+4}{x^3 + 4x}\, dx$
		\end{ex}
			\vs{1}
			\newpage
			
		\begin{ex}
			Evaluate $\ds \int \dfrac{x-1}{4x^2-4x+3}$
		\end{ex}
			\vs{1}
			\newpage
			
	\subsubsection*{Repeated Quadratic Factors}
	\addcontentsline{toc}{subsubsection}{Repeated Quadratic Factors}
		\begin{rmk}[Functions with Repeated Irreducible Quadratic Factors]
			Let $f(x) = \dfrac{P(x)}{Q(x)}$, where $\text{deg}(P) < \text{deg}(Q)$.  If $Q(x)$ has an irreducible quadratic factor of the form $(ax^2 + bx + c)^r$, then the decomposition of $f(x) = \dfrac{P(x)}{Q(x)}$ will have a term of the form \\ \\ \\
		\end{rmk}
		
		\begin{ex}
			Write the partial fraction decomposition for the function\\ $f(x) = \dfrac{x}{x(x+1)(x^2+x+1)(x^2+3)^3}$
		\end{ex}
			\vs{1}
			
		\begin{ex}
			Compute $\ds \int \dfrac{x^2+x+1}{(x^2+1)^2}\, dx$
		\end{ex}
			\vs{2}
			\newpage
			
		\begin{ex}
			Compute $\ds \int \dfrac{x^3+6x-2}{x^4+6x^2}\, dx$
		\end{ex}
			\vs{1}
			\newpage
			
		\begin{ex}
			Compute $\ds\int \dfrac{4x}{x^3 + x^2 + x + 1}\, dx$
		\end{ex}
			\vs{1}
			
		\begin{ex}
			Use substitution to evaluate $\ds \int \dfrac{dx}{x\sqrt{x-2}}$
		\end{ex}
			\vs{.5}
			\newpage
			
		\begin{ex}
			Use substitution to evaluate $\ds \int \dfrac{1}{(1+\sqrt{x})^2}\, dx$
		\end{ex}
			\vs{1}
			
		\begin{ex}
			Compute $\ds \int \dfrac{e^x}{(e^x-2)(e^{2x}+1)}\, dx$
		\end{ex}
			\vs{1}
			\newpage
			
			
	
	\subsection*{After Class Activities}
	\addcontentsline{toc}{subsection}{After Class Activities}
		\begin{ex}
			Evaluate the following integrals:
			\begin{enumerate}[(a)]
				\item $\ds \int_0^1 \dfrac{x-4}{x^2-5x+6}\, dx$
					\vs{1}
					
				\item $\ds \int \dfrac{1}{(t^2-1)^2}\, dt$
					\vs{1}
					\newpage
					
				\item $\ds \int \dfrac{10}{(x-1)(x^2+9)}\, dx$
					\vs{1}
					
				\item $\ds \int \dfrac{x^4 + 9x^2 + x + 2}{x^2 + 9}\, dx$
					\vs{1}
			\end{enumerate}
		\end{ex}
\clearpage
\end{document}