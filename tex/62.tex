\documentclass[notes]{subfiles}
\begin{document}
	\addcontentsline{toc}{section}{6.2 - Exponential Functions and Their Derivatives}
	\refstepcounter{section}
	\fancyhead[RO,LE]{\bfseries \nameref{cs62}} 
	\fancyhead[LO,RE]{\bfseries \small \currentname}
	\fancyfoot[C]{{}}
	\fancyfoot[RO,LE]{\large \thepage}	%Footer on Right \thepage is pagenumber
	\fancyfoot[LO,RE]{\large Chapter 6.2}
	
\section*{Exponential Functions \& Derivatives}\label{cs62}
	\subsection*{Before Class}
	\addcontentsline{toc}{subsection}{Before Class}
	\subsubsection*{Exponential Functions}
	\addcontentsline{toc}{subsubsection}{Exponential Functions}
		\showto{st}{
		\begin{defn}[Exponential Function]
			An \textbf{exponential function} is a function of the form \blank{2.5}, where $b$ is a positive constant.
		\end{defn}
		}
		
		\showto{ins}{
		\begin{defn}[Exponential Function]
			An \textbf{exponential function} is a function of the form $f(x) = b^x$, where $b$ is a positive constant.
		\end{defn}
		}
		
		\showto{st}{
		\begin{rmk}[Properties of Exponential Functions]
			Let $f(x) = b^x$.  Then, $f(x)$ has the following properties:\\ \\
			\begin{itemize}
				\setlength\itemsep{25pt}
				\item Domain: \blank{1.75}
				\item Range: \blank{1.75}
				\item If \blank{2}, then $f(x)$ is increasing
				\item If \blank{2}, then $f(x)$ is decreasing
				\item $\ds\lim_{x\to \infty} f(x) = \begin{cases}\blank{.75}& \text{if } \blank{1.5}\\ & \\ \blank{.75}& \text{if } \blank{1.5} \end{cases}$
				\item $\ds\lim_{x\to -\infty} f(x) =\begin{cases}\blank{.75}& \text{if } \blank{1.5}\\ & \\ \blank{.75}& \text{if } \blank{1.5} \end{cases}$ 	
			\end{itemize}
		\end{rmk}
		}
		
		\showto{ins}{
		\begin{rmk}[Properties of Exponential Functions]
			Let $f(x) = b^x$.  Then, $f(x)$ has the following properties:\\ \\
			\begin{itemize}
				\setlength\itemsep{10pt}
				\item Domain: $(-\infty,\infty)$
				\item Range: $(0,\infty)$
				\item If $b > 1$, then $f(x)$ is increasing
				\item If $0 < b < 1$, then $f(x)$ is decreasing
				\item $\ds\lim_{x\to \infty} f(x) = \begin{cases}\infty& \text{if } b > 1\\ & \\ 0 & \text{if } 0 < b < 1 \end{cases}$
				\item $\ds\lim_{x\to -\infty} f(x) = \begin{cases}0& \text{if } b > 1\\ & \\ \infty & \text{if } 0 < b < 1 \end{cases}$	
			\end{itemize}
		\end{rmk}
		}
		\newpage
		
		\begin{ex}
			For the following functions, find the limits and sketch the graph:
			\begin{enumerate}[(a)]
				\item $f(x) = 2(1.2^x) + 3$
					\vs{1}
					
				\item $g(x) = 3^{-x} - 1$
					\vs{1}
					
			\end{enumerate}
		\end{ex}
			
		\begin{defn}[Euler's Constant ($e$)]
			$e$ is defined to be the number for which $\ds \lim_{h\to 0} \dfrac{e^h-1}{h} = 1$
		\end{defn}
			\newpage
	\subsubsection*{Calculus of Exponentials}	
	\addcontentsline{toc}{subsubsection}{Calculus of Exponentials}
		\begin{rmk}[Derivative of an Exponential (First Attempt)]
			If $f(x) = b^x$, then $f'(x) = f'(0)b^x$
		\end{rmk}
		\begin{pf}
			\vs{2}
		\end{pf}
		
		This means we have the following interpretation of $f(x) = e^x$:
		\begin{rmk}[Special Meaning of $e$]
			$f(x) = e^x$ is the unique exponential function whose tangent line at the point $(0,1)$ is exactly 1, i.e. $f'(0) = 1$.
		\end{rmk}
		
		\begin{rmk}[Derivative of $e^x$]
			\[\dfrac{d}{dx}\left[e^x\right] = \]
		\end{rmk}
		
		\begin{rmk}[Antiderivative of $e^x$]
			\[\ds \int e^x\, dx = \]
		\end{rmk}
			\newpage
			
	\subsection*{Pre-Class Activities}
	\addcontentsline{toc}{subsection}{Pre-Class Activities}
		\begin{ex}
			Write the domain of the function:
			\begin{enumerate}[(a)]
				\item $f(x) = \dfrac{1-e^{x^2}}{1-e^{4-x^2}}$
					\vs{.5}
					
				\item $g(x) = \dfrac{1+x}{3^{\sin x}}$
					\vs{.5}
					
				\item $h(t) = \sqrt{4^t - 16}$
					\vs{.5}
					
			\end{enumerate}	
		\end{ex}
		
		\begin{ex}
			Find the indicated limit:
			\begin{enumerate}[(a)]
				\item $\ds \lim_{x\to \infty} (1.0001)^x$
					\vs{.5}
					
				\item $\ds \lim_{x\to \infty} \dfrac{e^{3x}-e^{-3x}}{e^{3x} + e^{-3x}}$
					\vs{.5}
					
				\item $\ds \lim_{x\to \infty} (e^{-2x}\sin x)$
					\vs{.5}
			\end{enumerate}
		\end{ex}
		\newpage
		
		\begin{ex}
			Find the derivative of the function:
			\begin{enumerate}[(a)]
				\item $f(x) = e^4$
					\vs{1}
					
				\item $g(r) = e^r + r^e$
					\vs{1}
					
				\item $f(x) = \dfrac{e^x}{1+e^x}$
					\vs{1}
			\end{enumerate}
		\end{ex}
		
		\begin{ex}
			Find the equation of the tangent line to the curve $y = xe^x$ at the point $(1,e)$.
		\end{ex}
			\vs{1}
			
		\begin{question}
			Use this space to write any questions or concerns you have from the pre-class portion of this section.
		\end{question}
			\vs{1}
			\newpage
			
	\subsection*{In Class}
	\addcontentsline{toc}{subsection}{In Class}
	\subsubsection*{Examples}
	\addcontentsline{toc}{subsubsection}{Examples}
		\begin{ex}
			Compute $f'(x)$, if $f(x) = e^{\tan x}$
		\end{ex}
			\vs{1}
			
		\begin{ex}
			Compute $f'(x)$, if $f(x) = \tan (e^x)$
		\end{ex}
			\vs{1}
			
		\begin{ex}
			Find $y'$ if $y = e^{-6x}\cos(2x)$
		\end{ex}
			\vs{1}
			
		\begin{ex}
			Find the absolute maximum and absolute minimum of $y = xe^{-x}$
		\end{ex}
			\vs{1}
			\newpage
			
		\begin{ex}
			Find $\dfrac{dy}{dx}$, if $e^{x/y} = y - x$
		\end{ex}
			\vs{1}
			
		\begin{ex}
			Compute the derivatives:
			\begin{enumerate}[(a)]
				\item $y = x^2e^{-1/x}$
					\vs{1}
					
				\item $g(x) = e^{x^2-x}$
					\vs{1}
					
				\item $f(t) = \sqrt{1+te^{-2t}}$
					\vs{1}
			\end{enumerate}
		\end{ex}
			\newpage
			
		\begin{ex}
			Find the absolute maximum and absolute minimum of $f(x) = xe^{-x^2/8}$ on $[-1,4]$
		\end{ex}
			\vs{1}
			
		\begin{ex}
			Evaluate the integral:
			\begin{enumerate}[(a)]
				\item $\ds \int_0^1 (x^e + e^x)\, dx$
					\vs{1}
					
				\item $\ds \int x^3e^{x^4}\, dx$
					\vs{1}
					
				\item $\ds \int e^x\sqrt{1+e^x}\, dx$
					\vs{1}
			\end{enumerate}
		\end{ex}	
			\newpage
			
		\begin{ex}
			Compute $\ds \int_1^2 \dfrac{e^{1/x}}{x^2}\, dx$
		\end{ex}
			\vs{1}
			
		\begin{ex}
			Find $f(x)$ if $f''(x) = 3e^x + 5\sin x$, $f(0) = 1$, and $f'(0) = 2$
		\end{ex}
			\vs{1}
			
		\begin{ex}
			The \emph{error function}, $\ds \text{erf}(x) = \dfrac{2}{\sqrt{\pi}}\int_0^x e^{-t^2}\,dt$ is a useful function in probability, statistics, and engineering.  Show that $\ds \int_a^b e^{-t^2}\, dt = \dfrac{1}{2}\sqrt{\pi}\left[\text{erf}(b) - \text{erf}(a)\right]$.
		\end{ex}
			\vs{1}
			\newpage
			
	\subsection*{After Class Activities}
	\addcontentsline{toc}{subsection}{After Class Activities}
		\begin{ex}
			Show that the function $y = e^x + e^{-x/2}$ satisfies the differential equation $2y'' - y' - y = 0$.
		\end{ex}
			\vs{1}
			
		\begin{ex}
			Find an equation of the tangent line to the curve $xe^y + ye^x = 1$ at the point $(0,1)$.
		\end{ex}
			\vs{1}
			
		\begin{ex}
			Compute $\dfrac{d^{1000}}{dx^{1000}}\left[xe^{-x}\right]$
		\end{ex}
			\vs{1}
			\newpage
		
		\begin{ex}
			If $f(x) = 3 + x + e^x$, find $(\inv{f})'(4)$
		\end{ex}
			\vs{1}
			
		\begin{ex}
			Evaluate $\ds \lim_{x\to \pi} \dfrac{e^{\sin x} - 1}{x-\pi}$
		\end{ex}
			\vs{1}
			
		\begin{ex}
			Find the volume of the solid obtained by rotating the region bounded by the curves $y = e^x$, $y = 0$, $x = 0$, and $x = 1$ about the $x-$axis.
		\end{ex}
			\vs{1}
				
\clearpage
\end{document}
